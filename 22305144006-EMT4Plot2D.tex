
\begin{eulernotebook}
\eulerheading{Menggambar Grafik 2D dengan EMT}
\begin{eulercomment}
Zainal Arrasyid Dalimunte\\
Matematika B-22305144006\\
\end{eulercomment}
\eulersubheading{}
\begin{eulercomment}
Notebook ini menjelaskan tentang cara menggambar berbagaikurva dan
grafik 2D dengan software EMT. EMT menyediakan fungsi plot2d() untuk
menggambar berbagai kurva dan grafik dua dimensi (2D).\\
\end{eulercomment}
\eulersubheading{Plot Dasar}
\begin{eulercomment}
Ada fungsi plot yang sangat mendasar. Ada koordinat layar, yang selalu
berkisar dari 0 hingga 1024 di setiap sumbu, tidak peduli apakah
layarnya persegi atau tidak. Terdapat koordinat plot, yang dapat
diatur dengan setplot(). Pemetaan antara koordinat tergantung pada
jendela plot saat ini. Sebagai contoh, default shrinkwindow()
menyisakan ruang untuk label sumbu dan judul plot.

Pada contoh, kita hanya menggambar beberapa garis acak dalam berbagai
warna. Untuk detail mengenai fungsi-fungsi ini, pelajari fungsi-fungsi
inti dari EMT.
\end{eulercomment}
\begin{eulerprompt}
>clg; // membersihkan layar
>window(0,0,1024,1024); // Menggunakan seluruh layar
>setplot(0,1,0,1); // mangatur koordinat plot
>hold on; // memulai mode timpa
>n=100; X=random(n,2); Y=random(n,2);  // mendapatkan nilai acak
>colors=rgb(random(n),random(n),random(n)); // mendapatkan warna acak
>loop 1 to n; color(colors[#]); plot(X[#],Y[#]); end; // plot
>hold off; // mengakhiri mode timpa
>insimg; // memasukkan ke notebook
\end{eulerprompt}
\eulerimg{27}{images/22305144006-EMT4Plot2D-001.png}
\begin{eulerprompt}
>reset;
\end{eulerprompt}
\begin{eulercomment}
Anda harus menahan grafik, karena perintah plot() akan menghapus
jendela plot.

Untuk menghapus semua yang telah kita lakukan, kita gunakan reset().

Untuk menampilkan gambar hasil plot di layar notebook, perintah
plot2d() dapat diakhiri dengan titik dua (:). Cara lain adalah
perintah plot2d() diakhiri dengan titik koma (;), kemudian gunakan
perintah insimg() untuk menampilkan gambar hasil plot.

Sebagai contoh lain, kita menggambar sebuah plot sebagai inset pada
plot yang lain. Hal ini dilakukan dengan mendefinisikan jendela plot
yang lebih kecil. Perhatikan bahwa jendela ini tidak menyediakan ruang
untuk label sumbu di luar jendela plot. Kita harus menambahkan
beberapa margin untuk hal ini sesuai kebutuhan. Perhatikan bahwa kita
menyimpan dan mengembalikan jendela penuh, dan menahan plot saat ini
ketika kita membuat inset.
\end{eulercomment}
\begin{eulerprompt}
>plot2d("x^3-x");
>xw=200; yw=100; ww=300; hw=300;
>ow=window();
>window(xw,yw,xw+ww,yw+hw);
>hold on;
>barclear(xw-50,yw-10,ww+60,ww+60);
>plot2d("x^4-x",grid=6):
\end{eulerprompt}
\eulerimg{27}{images/22305144006-EMT4Plot2D-002.png}
\begin{eulerprompt}
>hold off;
>window(ow);
\end{eulerprompt}
\begin{eulercomment}
Plot dengan beberapa angka dicapai dengan cara yang sama. Ada fungsi
kegunaan figure() untuk ini.

\end{eulercomment}
\eulersubheading{Aspek Plot}
\begin{eulercomment}
Plot default menggunakan jendela plot persegi. Anda dapat mengubahnya
dengan fungsi aspect(). Jangan lupa untuk mengatur ulang aspeknya
nanti. Anda juga dapat mengubah default ini di menu dengan "Set
Aspect" ke rasio aspek tertentu atau ke ukuran jendela grafik saat
ini.

Tetapi Anda juga dapat mengubahnya untuk satu plot. Untuk ini, ukuran
area plot saat ini diubah, dan jendela diatur sedemikian rupa sehingga
label memiliki ruang yang cukup.
\end{eulercomment}
\begin{eulerprompt}
>aspect(2); // rasio panjang dan lebar 2:1
>plot2d(["sin(x)","cos(x)"],0,2pi):
\end{eulerprompt}
\eulerimg{13}{images/22305144006-EMT4Plot2D-003.png}
\begin{eulerprompt}
>aspect();
>reset;
\end{eulerprompt}
\begin{eulercomment}
Fungsi reset () mengembalikan default plot, termasuk rasio aspek.\\
\begin{eulercomment}
\eulerheading{Plot 2D di Euler}
\begin{eulercomment}
EMT Math Toolbox memiliki plot dalam bentuk 2D, baik untuk data maupun
fungsi. EMT menggunakan fungsi plot2d. Fungsi ini dapat memplot fungsi
dan data.

Dimungkinkan untuk memplot di Maxima menggunakan Gnuplot atau di
Python menggunakan Math Plot Lib.

Euler dapat memplot plot 2D dari

- ekspresi\\
- fungsi, variabel, atau kurva berparameter,\\
- vektor nilai x-y,\\
- awan titik-titik di bidang,\\
- kurva implisit dengan level atau wilayah level.\\
- Fungsi yang kompleks

Gaya plot mencakup berbagai gaya untuk garis dan titik, plot batang,
dan plot berbayang.\\
\begin{eulercomment}
\eulerheading{Plot Ekspresi atau Variabel}
\begin{eulercomment}
Ekspresi tunggal dalam "x" (misalnya "4*x\textasciicircum{}2") atau nama fungsi
(misalnya "f") menghasilkan grafik fungsi.

Berikut ini adalah contoh paling dasar, yang menggunakan rentang
default dan menetapkan rentang y yang tepat agar sesuai dengan plot
fungsi.

Catatan: Jika Anda mengakhiri baris perintah dengan tanda titik dua
":", plot akan disisipkan ke dalam jendela teks. Jika tidak, tekan TAB
untuk melihat plot jika jendela plot tertutup.
\end{eulercomment}
\begin{eulerprompt}
>plot2d("x^2"):
\end{eulerprompt}
\eulerimg{27}{images/22305144006-EMT4Plot2D-004.png}
\begin{eulerprompt}
>aspect(1.5); plot2d("x^3-x"):
\end{eulerprompt}
\eulerimg{17}{images/22305144006-EMT4Plot2D-005.png}
\begin{eulerprompt}
>a:=5.6; plot2d("exp(-a*x^2)/a"); insimg(30); // menampilkan gambar hasil plot setinggi 25 baris
\end{eulerprompt}
\eulerimg{17}{images/22305144006-EMT4Plot2D-006.png}
\begin{eulercomment}
Dari beberapa contoh sebelumnya Anda dapat melihat bahwa aslinya
gambar plot menggunakan sumbu X dengan rentang nilai dari -2 sampai
dengan 2. Untuk mengubah rentang nilai X dan Y, Anda dapat menambahkan
nilai-nilai batas X (dan Y) di belakang ekspresi yang digambar.

Rentang plot ditetapkan dengan parameter yang ditetapkan berikut ini

- a,b: rentang x (default -2,2)\\
- c, d: rentang y (default: skala dengan nilai)\\
- r: sebagai alternatif, radius di sekitar pusat plot\\
- cx, cy: koordinat pusat plot (default 0,0)
\end{eulercomment}
\begin{eulerprompt}
>plot2d("x^3-x",-1,2):
\end{eulerprompt}
\eulerimg{17}{images/22305144006-EMT4Plot2D-007.png}
\begin{eulerprompt}
>plot2d("sin(x)",-2*pi,2*pi): // plot sin(x) pada interval [-2pi, 2pi]
\end{eulerprompt}
\eulerimg{17}{images/22305144006-EMT4Plot2D-008.png}
\begin{eulerprompt}
>plot2d("cos(x)","sin(3*x)",xmin=0,xmax=2pi):
\end{eulerprompt}
\eulerimg{17}{images/22305144006-EMT4Plot2D-009.png}
\begin{eulercomment}
Alternatif untuk tanda titik dua adalah perintah insimg(lines), yang
menyisipkan plot yang menempati sejumlah baris teks tertentu.

Dalam opsi, plot dapat diatur untuk muncul

- di jendela terpisah yang dapat diubah ukurannya,\\
- di jendela buku catatan.

Lebih banyak gaya yang dapat dicapai dengan perintah plot tertentu.

Dalam hal apa pun, tekan tombol tabulator untuk melihat plot, jika
disembunyikan.

Untuk membagi jendela menjadi beberapa plot, gunakan perintah
figure(). Pada contoh, kita memplot x\textasciicircum{}1 sampai x\textasciicircum{}4 ke dalam 4 bagian
jendela. figure(0) akan mereset jendela default.
\end{eulercomment}
\begin{eulerprompt}
>reset;
>figure(2,2); ...
>for n=1 to 4; figure(n); plot2d("x^"+n); end; ...
>figure(0):
\end{eulerprompt}
\eulerimg{27}{images/22305144006-EMT4Plot2D-010.png}
\begin{eulercomment}
Pada plot2d(), terdapat beberapa gaya alternatif yang tersedia dengan
grid=x. Sebagai gambaran umum, kami menampilkan berbagai gaya grid
dalam satu gambar (lihat di bawah ini untuk perintah figure()). Gaya
grid=0 tidak disertakan. Gaya ini tidak menampilkan grid dan frame.
\end{eulercomment}
\begin{eulerprompt}
>figure(3,3); ...
>for k=1:9; figure(k); plot2d("x^3-x",-2,1,grid=k); end; ...
>figure(0):
\end{eulerprompt}
\eulerimg{27}{images/22305144006-EMT4Plot2D-011.png}
\begin{eulercomment}
Jika argumen untuk plot2d() adalah sebuah ekspresi yang diikuti oleh
empat angka, angka-angka ini adalah rentang x dan y untuk plot.

Atau, a, b, c, d dapat ditentukan sebagai parameter yang ditetapkan
sebagai a=... dst.

Pada contoh berikut, kita mengubah gaya grid, menambahkan label, dan
menggunakan label vertikal untuk sumbu y.
\end{eulercomment}
\begin{eulerprompt}
>aspect(1.5); plot2d("sin(x)",0,2pi,-1.2,1.2,grid=5,xl="x",yl="sin(x)"):
\end{eulerprompt}
\eulerimg{17}{images/22305144006-EMT4Plot2D-012.png}
\begin{eulerprompt}
>plot2d("sin(x)+cos(2*x)",0,4pi):
\end{eulerprompt}
\eulerimg{17}{images/22305144006-EMT4Plot2D-013.png}
\begin{eulercomment}
Gambar yang dihasilkan dengan menyisipkan plot ke dalam jendela teks
disimpan dalam direktori yang sama dengan notebook, secara default
dalam subdirektori bernama "images". Gambar-gambar tersebut juga
digunakan oleh ekspor HTML.

Anda cukup menandai gambar mana saja dan menyalinnya ke clipboard
dengan Ctrl-C. Tentu saja, Anda juga dapat mengekspor grafik saat ini
dengan fungsi-fungsi pada menu File.

Fungsi atau ekspresi dalam plot2d dievaluasi secara adaptif. Untuk
kecepatan yang lebih tinggi, matikan plot adaptif dengan \textless{}adaptive dan
tentukan jumlah subinterval dengan n=... Hal ini hanya diperlukan pada
kasus-kasus yang jarang terjadi.
\end{eulercomment}
\begin{eulerprompt}
>plot2d("sign(x)*exp(-x^2)",-1,1,<adaptive,n=10000):
\end{eulerprompt}
\eulerimg{17}{images/22305144006-EMT4Plot2D-014.png}
\begin{eulerprompt}
>plot2d("x^x",r=1.2,cx=1,cy=1):
\end{eulerprompt}
\eulerimg{17}{images/22305144006-EMT4Plot2D-015.png}
\begin{eulercomment}
Perhatikan bahwa x\textasciicircum{}x tidak didefinisikan untuk x\textless{}=0. Fungsi plot2d
menangkap kesalahan ini, dan mulai memplot segera setelah fungsi
didefinisikan. Hal ini berlaku untuk semua fungsi yang mengembalikan
NAN di luar jangkauan definisinya.
\end{eulercomment}
\begin{eulerprompt}
>plot2d("log(x)",-0.1,2):
\end{eulerprompt}
\eulerimg{17}{images/22305144006-EMT4Plot2D-016.png}
\begin{eulercomment}
Parameter square=true (atau \textgreater{}square) memilih rentang y secara otomatis
sehingga hasilnya adalah jendela plot persegi. Perhatikan bahwa secara
default, Euler menggunakan ruang persegi di dalam jendela plot.
\end{eulercomment}
\begin{eulerprompt}
>plot2d("x^3-x",>square):
\end{eulerprompt}
\eulerimg{17}{images/22305144006-EMT4Plot2D-017.png}
\begin{eulerprompt}
>plot2d(''integrate("sin(x)*exp(-x^2)",0,x)'',0,2): // plot integral
\end{eulerprompt}
\eulerimg{17}{images/22305144006-EMT4Plot2D-018.png}
\begin{eulercomment}
Jika Anda membutuhkan lebih banyak ruang untuk label-y, panggil
shrinkwindow() dengan parameter lebih kecil, atau tetapkan nilai
positif untuk "smaller" pada plot2d().
\end{eulercomment}
\begin{eulerprompt}
>plot2d("gamma(x)",1,10,yl="y-values",smaller=6,<vertical):
\end{eulerprompt}
\eulerimg{17}{images/22305144006-EMT4Plot2D-019.png}
\begin{eulercomment}
Ekspresi simbolik juga dapat digunakan, karena disimpan sebagai
ekspresi string sederhana.
\end{eulercomment}
\begin{eulerprompt}
>x=linspace(0,2pi,1000); plot2d(sin(5x),cos(7x)):
\end{eulerprompt}
\eulerimg{17}{images/22305144006-EMT4Plot2D-020.png}
\begin{eulerprompt}
>a:=5.6; expr &= exp(-a*x^2)/a; // mendefinisikan ekspresi
>plot2d(expr,-2,2): // plot dari -2 ke 2
\end{eulerprompt}
\eulerimg{17}{images/22305144006-EMT4Plot2D-021.png}
\begin{eulerprompt}
>plot2d(expr,r=1,thickness=2): // plot dalam persegi di sekitar (0,0)
\end{eulerprompt}
\eulerimg{17}{images/22305144006-EMT4Plot2D-022.png}
\begin{eulerprompt}
>plot2d(&diff(expr,x),>add,style="--",color=red): // menambah plot lain
\end{eulerprompt}
\eulerimg{17}{images/22305144006-EMT4Plot2D-023.png}
\begin{eulerprompt}
>plot2d(&diff(expr,x,2),a=-2,b=2,c=-2,d=1): // plot dalam persegi panjang
\end{eulerprompt}
\eulerimg{17}{images/22305144006-EMT4Plot2D-024.png}
\begin{eulerprompt}
>plot2d(&diff(expr,x),a=-2,b=2,>square): // menjaga plot tetap persegi
\end{eulerprompt}
\eulerimg{17}{images/22305144006-EMT4Plot2D-025.png}
\begin{eulerprompt}
>plot2d("x^2",0,1,steps=1,color=red,n=10):
\end{eulerprompt}
\eulerimg{17}{images/22305144006-EMT4Plot2D-026.png}
\begin{eulerprompt}
>plot2d("x^2",>add,steps=2,color=blue,n=10):
\end{eulerprompt}
\eulerimg{17}{images/22305144006-EMT4Plot2D-027.png}
\eulerheading{Fungsi dalam satu Parameter}
\begin{eulercomment}
Fungsi plot yang paling penting untuk plot planar adalah plot2d().
Fungsi ini diimplementasikan dalam bahasa Euler dalam file "plot.e",
yang dimuat pada awal program.

Berikut adalah beberapa contoh penggunaan fungsi. Seperti biasa dalam
EMT, fungsi yang bekerja untuk fungsi atau ekspresi lain, Anda dapat
mengoper parameter tambahan (selain x) yang bukan variabel global ke
fungsi dengan parameter titik koma atau dengan koleksi panggilan.
\end{eulercomment}
\begin{eulerprompt}
>function f(x,a) := x^2/a+a*x^2-x; // mendefiniskan fungsi
>a=0.3; plot2d("f",0,1;a): // plot dengan a=0.3
\end{eulerprompt}
\eulerimg{17}{images/22305144006-EMT4Plot2D-028.png}
\begin{eulerprompt}
>plot2d("f",0,1;0.4): // plot dengan a=0.4
\end{eulerprompt}
\eulerimg{17}{images/22305144006-EMT4Plot2D-029.png}
\begin{eulerprompt}
>plot2d(\{\{"f",0.2\}\},0,1): // plot dengan a=0.2
\end{eulerprompt}
\eulerimg{17}{images/22305144006-EMT4Plot2D-030.png}
\begin{eulerprompt}
>plot2d(\{\{"f(x,b)",b=0.1\}\},0,1): // plot dengan 0.1
\end{eulerprompt}
\eulerimg{17}{images/22305144006-EMT4Plot2D-031.png}
\begin{eulerprompt}
>function f(x) := x^3-x; ...
>plot2d("f",r=1):
\end{eulerprompt}
\eulerimg{17}{images/22305144006-EMT4Plot2D-032.png}
\begin{eulercomment}
Berikut ini adalah ringkasan dari fungsi yang diterima

- ekspresi atau ekspresi simbolik dalam x\\
- fungsi atau fungsi simbolik dengan nama sebagai "f"\\
- fungsi-fungsi simbolik hanya dengan nama f

Fungsi plot2d() juga menerima fungsi simbolik. Untuk fungsi simbolik,
nama saja sudah cukup.
\end{eulercomment}
\begin{eulerprompt}
>function f(x) &= diff(x^x,x)
\end{eulerprompt}
\begin{euleroutput}
  
                              x
                             x  (log(x) + 1)
  
\end{euleroutput}
\begin{eulerprompt}
>plot2d(f,0,2):
\end{eulerprompt}
\eulerimg{17}{images/22305144006-EMT4Plot2D-033.png}
\begin{eulercomment}
Tentu saja, untuk ekspresi atau ungkapan simbolik, nama variabel sudah
cukup untuk memplotnya.
\end{eulercomment}
\begin{eulerprompt}
>expr &= sin(x)*exp(-x)
\end{eulerprompt}
\begin{euleroutput}
  
                                - x
                               E    sin(x)
  
\end{euleroutput}
\begin{eulerprompt}
>plot2d(expr,0,3pi):
\end{eulerprompt}
\eulerimg{17}{images/22305144006-EMT4Plot2D-034.png}
\begin{eulerprompt}
>function f(x) &= x^x;
>plot2d(f,r=1,cx=1,cy=1,color=blue,thickness=2);
>plot2d(&diff(f(x),x),>add,color=red,style="-.-",thickness=2):
\end{eulerprompt}
\eulerimg{17}{images/22305144006-EMT4Plot2D-035.png}
\begin{eulercomment}
Untuk gaya garis, ada berbagai pilihan.

- style = "...". Pilih dari "-", "--", "-.", ".", ".-.", "-.-".\\
- color: Lihat di bawah untuk warna.\\
- ketebalan: Default adalah 1.

Warna dapat dipilih sebagai salah satu warna default, atau sebagai
warna RGB.

- 0..15: indeks warna default.\\
- konstanta warna: putih, hitam, merah, hijau, biru, cyan, zaitun,
abu-abu muda, abu-abu, abu-abu tua, oranye, hijau muda, pirus, biru
muda, oranye muda, kuning\\
- rgb (merah, hijau, biru): parameter adalah real dalam [0,1].
\end{eulercomment}
\begin{eulerprompt}
>plot2d("exp(-x^2)",r=2,color=13,thickness=3,style="--"):
\end{eulerprompt}
\eulerimg{13}{images/22305144006-EMT4Plot2D-036.png}
\begin{eulercomment}
Berikut ini adalah pemandangan warna EMT yang sudah ditetapkan
sebelumnya.
\end{eulercomment}
\begin{eulerprompt}
>aspect(2); columnsplot(ones(1,16),lab=0:15,grid=0,color=0:15):
\end{eulerprompt}
\eulerimg{13}{images/22305144006-EMT4Plot2D-037.png}
\begin{eulercomment}
Tetapi Anda bisa menggunakan warna apa saja.
\end{eulercomment}
\begin{eulerprompt}
>columnsplot(ones(1,16),grid=0,color=rgb(0,0,linspace(0,1,15))):
\end{eulerprompt}
\eulerimg{13}{images/22305144006-EMT4Plot2D-038.png}
\eulerheading{Menggambar beberapa kurva pada bidang koordinat yang sama}
\begin{eulercomment}
Memplot lebih dari satu fungsi (beberapa fungsi) ke dalam satu jendela
dapat dilakukan dengan berbagai cara. Salah satu caranya adalah dengan
menggunakan \textgreater{}add untuk beberapa pemanggilan ke plot2d secara
bersamaan, kecuali pemanggilan pertama. Kita telah menggunakan fitur
ini pada contoh di atas.
\end{eulercomment}
\begin{eulerprompt}
>aspect(); plot2d("cos(x)",r=2,grid=6); plot2d("x",style=".",>add):
\end{eulerprompt}
\eulerimg{27}{images/22305144006-EMT4Plot2D-039.png}
\begin{eulerprompt}
>aspect(1.5); plot2d("sin(x)",0,2pi); plot2d("cos(x)",color=blue,style="--",>add):
\end{eulerprompt}
\eulerimg{17}{images/22305144006-EMT4Plot2D-040.png}
\begin{eulercomment}
Salah satu kegunaan \textgreater{}add adalah untuk menambahkan titik pada kurva.
\end{eulercomment}
\begin{eulerprompt}
>plot2d("sin(x)",0,pi); plot2d(2,sin(2),>points,>add):
\end{eulerprompt}
\eulerimg{17}{images/22305144006-EMT4Plot2D-041.png}
\begin{eulercomment}
Kami menambahkan titik perpotongan dengan label (pada posisi "cl"
untuk kiri tengah), dan menyisipkan hasilnya ke dalam buku catatan.
Kami juga menambahkan judul ke plot.
\end{eulercomment}
\begin{eulerprompt}
>plot2d(["cos(x)","x"],r=1.1,cx=0.5,cy=0.5, ...
>  color=[black,blue],style=["-","."], ...
>  grid=1);
>x0=solve("cos(x)-x",1);  ...
>  plot2d(x0,x0,>points,>add,title="Intersection Demo");  ...
>  label("cos(x) = x",x0,x0,pos="cl",offset=20):
\end{eulerprompt}
\eulerimg{17}{images/22305144006-EMT4Plot2D-042.png}
\begin{eulercomment}
Dalam demo berikut ini, kami memplot fungsi sinc(x)=sin(x)/x dan
ekspansi Taylor ke-8 dan ke-16. Kami menghitung ekspansi ini
menggunakan Maxima melalui ekspresi simbolik.\\
Plot ini dilakukan dalam perintah multi-baris berikut dengan tiga
pemanggilan plot2d(). Perintah kedua dan ketiga memiliki set flag
\textgreater{}add, yang membuat plot menggunakan rentang sebelumnya.

Kami menambahkan sebuah kotak label yang menjelaskan fungsi-fungsi
tersebut.
\end{eulercomment}
\begin{eulerprompt}
>$taylor(sin(x)/x,x,0,4)
\end{eulerprompt}
\begin{eulerformula}
\[
\frac{x^4}{120}-\frac{x^2}{6}+1
\]
\end{eulerformula}
\begin{eulerprompt}
>plot2d("sinc(x)",0,4pi,color=green,thickness=2); ...
>  plot2d(&taylor(sin(x)/x,x,0,8),>add,color=blue,style="--"); ...
>  plot2d(&taylor(sin(x)/x,x,0,16),>add,color=red,style="-.-"); ...
>  labelbox(["sinc","T8","T16"],styles=["-","--","-.-"], ...
>    colors=[black,blue,red]):
\end{eulerprompt}
\eulerimg{17}{images/22305144006-EMT4Plot2D-044.png}
\begin{eulercomment}
Pada contoh berikut, kami menghasilkan Polinomial Bernstein.

\end{eulercomment}
\begin{eulerformula}
\[
B_i(x) = \binom{n}{i} x^i (1-x)^{n-i}
\]
\end{eulerformula}
\begin{eulerprompt}
>plot2d("(1-x)^10",0,1); // plot first function
>for i=1 to 10; plot2d("bin(10,i)*x^i*(1-x)^(10-i)",>add); end;
>insimg;
\end{eulerprompt}
\eulerimg{17}{images/22305144006-EMT4Plot2D-046.png}
\begin{eulercomment}
Metode kedua menggunakan sepasang matriks nilai x dan matriks nilai y
dengan ukuran yang sama.

Kita membuat sebuah matriks nilai dengan satu Polinomial Bernstein di
setiap baris. Untuk ini, kita cukup menggunakan vektor kolom i.
Lihatlah pengantar tentang bahasa matriks untuk mempelajari lebih
lanjut.
\end{eulercomment}
\begin{eulerprompt}
>x=linspace(0,1,500);
>n=10; k=(0:n)'; // n is row vector, k is column vector
>y=bin(n,k)*x^k*(1-x)^(n-k); // y is a matrix then
>plot2d(x,y):
\end{eulerprompt}
\eulerimg{17}{images/22305144006-EMT4Plot2D-047.png}
\begin{eulercomment}
Perhatikan bahwa parameter warna dapat berupa vektor. Kemudian setiap
warna digunakan untuk setiap baris matriks.
\end{eulercomment}
\begin{eulerprompt}
>x=linspace(0,1,200); y=x^(1:10)'; plot2d(x,y,color=1:10):
\end{eulerprompt}
\eulerimg{17}{images/22305144006-EMT4Plot2D-048.png}
\begin{eulercomment}
Metode lainnya adalah menggunakan vektor ekspresi (string). Anda
kemudian dapat menggunakan larik warna, larik gaya, dan larik
ketebalan dengan panjang yang sama.
\end{eulercomment}
\begin{eulerprompt}
>plot2d(["sin(x)","cos(x)"],0,2pi,color=4:5): 
\end{eulerprompt}
\eulerimg{17}{images/22305144006-EMT4Plot2D-049.png}
\begin{eulerprompt}
>plot2d(["sin(x)","cos(x)"],0,2pi): // plot vector of expressions
\end{eulerprompt}
\eulerimg{17}{images/22305144006-EMT4Plot2D-050.png}
\begin{eulercomment}
Kita bisa mendapatkan vektor seperti itu dari Maxima dengan
menggunakan makelist() dan mxm2str().
\end{eulercomment}
\begin{eulerprompt}
>v &= makelist(binomial(10,i)*x^i*(1-x)^(10-i),i,0,10) // make list
\end{eulerprompt}
\begin{euleroutput}
  
                 10            9              8  2             7  3
         [(1 - x)  , 10 (1 - x)  x, 45 (1 - x)  x , 120 (1 - x)  x , 
             6  4             5  5             4  6             3  7
  210 (1 - x)  x , 252 (1 - x)  x , 210 (1 - x)  x , 120 (1 - x)  x , 
            2  8              9   10
  45 (1 - x)  x , 10 (1 - x) x , x  ]
  
\end{euleroutput}
\begin{eulerprompt}
>mxm2str(v) // get a vector of strings from the symbolic vector
\end{eulerprompt}
\begin{euleroutput}
  (1-x)^10
  10*(1-x)^9*x
  45*(1-x)^8*x^2
  120*(1-x)^7*x^3
  210*(1-x)^6*x^4
  252*(1-x)^5*x^5
  210*(1-x)^4*x^6
  120*(1-x)^3*x^7
  45*(1-x)^2*x^8
  10*(1-x)*x^9
  x^10
\end{euleroutput}
\begin{eulerprompt}
>plot2d(mxm2str(v),0,1): // plot functions
\end{eulerprompt}
\eulerimg{17}{images/22305144006-EMT4Plot2D-051.png}
\begin{eulercomment}
Alternatif lain adalah dengan menggunakan bahasa matriks Euler.

Jika sebuah ekspresi menghasilkan sebuah matriks fungsi, dengan satu
fungsi di setiap baris, semua fungsi ini akan diplot ke dalam satu
plot.

Untuk ini, gunakan vektor parameter dalam bentuk vektor kolom. Jika
sebuah larik warna ditambahkan, maka akan digunakan untuk setiap baris
plot.
\end{eulercomment}
\begin{eulerprompt}
>n=(1:10)'; plot2d("x^n",0,1,color=1:10):
\end{eulerprompt}
\eulerimg{17}{images/22305144006-EMT4Plot2D-052.png}
\begin{eulercomment}
Ekspresi dan fungsi satu baris dapat melihat variabel global.

Jika Anda tidak dapat menggunakan variabel global, Anda perlu
menggunakan fungsi dengan parameter tambahan, dan memberikan parameter
ini sebagai parameter titik koma.

Berhati-hatilah untuk meletakkan semua parameter yang diberikan di
akhir perintah plot2d. Pada contoh ini kita mengoper a=5 ke fungsi f,
yang kita plot dari -10 ke 10.
\end{eulercomment}
\begin{eulerprompt}
>function f(x,a) := 1/a*exp(-x^2/a); ...
>plot2d("f",-10,10;5,thickness=2,title="a=5"):
\end{eulerprompt}
\eulerimg{17}{images/22305144006-EMT4Plot2D-053.png}
\begin{eulercomment}
Atau, gunakan koleksi dengan nama fungsi dan semua parameter tambahan.
Daftar khusus ini disebut koleksi panggilan, dan itu adalah cara yang
lebih disukai untuk mengoper argumen ke fungsi yang dengan sendirinya
dioper sebagai argumen ke fungsi lain.

Pada contoh berikut, kita menggunakan perulangan untuk memplot
beberapa fungsi (lihat tutorial tentang pemrograman perulangan).
\end{eulercomment}
\begin{eulerprompt}
>plot2d(\{\{"f",1\}\},-10,10); ...
>for a=2:10; plot2d(\{\{"f",a\}\},>add); end:
\end{eulerprompt}
\eulerimg{17}{images/22305144006-EMT4Plot2D-054.png}
\begin{eulercomment}
Kita dapat mencapai hasil yang sama dengan cara berikut menggunakan
bahasa matriks EMT. Setiap baris dari matriks f(x,a) adalah satu
fungsi. Selain itu, kita dapat mengatur warna untuk setiap baris
matriks. Klik dua kali pada fungsi getspectral() untuk penjelasannya.
\end{eulercomment}
\begin{eulerprompt}
>x=-10:0.01:10; a=(1:10)'; plot2d(x,f(x,a),color=getspectral(a/10)):
\end{eulerprompt}
\eulerimg{17}{images/22305144006-EMT4Plot2D-055.png}
\eulersubheading{Label Teks}
\begin{eulercomment}
Dekorasi sederhana dapat berupa

- sebuah judul dengan title="..."\\
- label x dan y dengan xl="...", yl="..."\\
- label teks lain dengan label("...",x,y)

Perintah label akan memplotkan ke dalam plot saat ini pada koordinat
plot (x,y). Perintah ini dapat menerima sebuah argumen posisi.
\end{eulercomment}
\begin{eulerprompt}
>plot2d("x^3-x",-1,2,title="y=x^3-x",yl="y",xl="x"):
\end{eulerprompt}
\eulerimg{17}{images/22305144006-EMT4Plot2D-056.png}
\begin{eulerprompt}
>expr := "log(x)/x"; ...
>  plot2d(expr,0.5,5,title="y="+expr,xl="x",yl="y"); ...
>  label("(1,0)",1,0); label("Max",E,expr(E),pos="lc"):
\end{eulerprompt}
\eulerimg{17}{images/22305144006-EMT4Plot2D-057.png}
\begin{eulercomment}
Ada juga fungsi labelbox(), yang dapat menampilkan fungsi dan teks.
Fungsi ini membutuhkan vektor string dan warna, satu item untuk setiap
fungsi.
\end{eulercomment}
\begin{eulerprompt}
>function f(x) &= x^2*exp(-x^2);  ...
>plot2d(&f(x),a=-3,b=3,c=-1,d=1);  ...
>plot2d(&diff(f(x),x),>add,color=blue,style="--"); ...
>labelbox(["function","derivative"],styles=["-","--"], ...
>   colors=[black,blue],w=0.4):
\end{eulerprompt}
\eulerimg{17}{images/22305144006-EMT4Plot2D-058.png}
\begin{eulercomment}
Kotak tersebut berlabuh di kanan atas secara default, tetapi \textgreater{}left
menambatkannya di kiri atas. Anda dapat memindahkannya ke tempat mana
pun yang Anda suka. Posisi jangkar adalah sudut kanan atas kotak, dan
angkanya adalah pecahan dari ukuran jendela grafik. Lebarnya otomatis.

Untuk plot titik, kotak label juga dapat digunakan. Tambahkan
parameter \textgreater{}point, atau vektor bendera, satu untuk setiap label.

Pada contoh berikut, hanya ada satu fungsi. Jadi kita dapat
menggunakan string sebagai pengganti vektor string. Kita mengatur
warna teks menjadi hitam untuk contoh ini.
\end{eulercomment}
\begin{eulerprompt}
>n=10; plot2d(0:n,bin(n,0:n),>addpoints); ...
>labelbox("Binomials",styles="[]",>points,x=0.1,y=0.1, ...
>tcolor=5,>left):
\end{eulerprompt}
\eulerimg{17}{images/22305144006-EMT4Plot2D-059.png}
\begin{eulercomment}
Gaya plot ini juga tersedia di statplot(). Seperti pada plot2d() warna
dapat diatur untuk setiap baris plot. Terdapat lebih banyak plot
khusus untuk keperluan statistik (lihat tutorial tentang statistik).
\end{eulercomment}
\begin{eulerprompt}
>statplot(1:10,random(2,10),color=[red,blue]):
\end{eulerprompt}
\eulerimg{17}{images/22305144006-EMT4Plot2D-060.png}
\begin{eulercomment}
Fitur yang serupa adalah fungsi textbox().

Lebarnya secara default adalah lebar maksimal baris teks. Tetapi bisa
juga diatur oleh pengguna.
\end{eulercomment}
\begin{eulerprompt}
>function f(x) &= exp(-x)*sin(2*pi*x); ...
>plot2d("f(x)",0,2pi); ...
>textbox(latex("\(\backslash\)text\{Example of a damped oscillation\}\(\backslash\) f(x)=e^\{-x\}sin(2\(\backslash\)pi x)"),w=0.85):
\end{eulerprompt}
\eulerimg{17}{images/22305144006-EMT4Plot2D-061.png}
\begin{eulercomment}
Label teks, judul, kotak label, dan teks lainnya dapat berisi string
Unicode (lihat sintaks EMT untuk mengetahui lebih lanjut tentang
string Unicode).
\end{eulercomment}
\begin{eulerprompt}
>plot2d("x^3-x",title=u"x &rarr; x&sup3; - x"):
\end{eulerprompt}
\eulerimg{17}{images/22305144006-EMT4Plot2D-062.png}
\begin{eulercomment}
Label pada sumbu x dan y bisa vertikal, begitu juga dengan sumbu.
\end{eulercomment}
\begin{eulerprompt}
>plot2d("sinc(x)",0,2pi,xl="x",yl=u"x &rarr; sinc(x)",>vertical):
\end{eulerprompt}
\eulerimg{17}{images/22305144006-EMT4Plot2D-063.png}
\eulersubheading{LaTeX}
\begin{eulercomment}
Anda juga dapat memplot formula LaTeX jika Anda telah menginstal
sistem LaTeX. Saya merekomendasikan MiKTeX. Jalur ke binari "latex"
dan "dvipng" harus berada di jalur sistem, atau Anda harus mengatur
LaTeX di menu opsi.

Perhatikan, bahwa penguraian LaTeX berjalan lambat. Jika Anda ingin
menggunakan LaTeX dalam plot animasi, Anda harus memanggil latex()
sebelum perulangan satu kali dan menggunakan hasilnya (gambar dalam
matriks RGB).

Pada plot berikut ini, kita menggunakan LaTeX untuk label x dan y,
sebuah label, kotak label dan judul plot.
\end{eulercomment}
\begin{eulerprompt}
>plot2d("exp(-x)*sin(x)/x",a=0,b=2pi,c=0,d=1,grid=6,color=blue, ...
>  title=latex("\(\backslash\)text\{Function $\(\backslash\)Phi$\}"), ...
>  xl=latex("\(\backslash\)phi"),yl=latex("\(\backslash\)Phi(\(\backslash\)phi)")); ...
>textbox( ...
>  latex("\(\backslash\)Phi(\(\backslash\)phi) = e^\{-\(\backslash\)phi\} \(\backslash\)frac\{\(\backslash\)sin(\(\backslash\)phi)\}\{\(\backslash\)phi\}"),x=0.8,y=0.5); ...
>label(latex("\(\backslash\)Phi",color=blue),1,0.4):
\end{eulerprompt}
\eulerimg{17}{images/22305144006-EMT4Plot2D-064.png}
\begin{eulercomment}
Seringkali, kita menginginkan spasi dan label teks yang tidak sesuai
pada sumbu x. Kita dapat menggunakan xaxis() dan yaxis() seperti yang
akan kita tunjukkan nanti.

Cara termudah adalah dengan membuat plot kosong dengan sebuah frame
menggunakan grid=4, dan kemudian menambahkan grid dengan ygrid() dan
xgrid(). Pada contoh berikut, kita menggunakan tiga buah string LaTeX
untuk label pada sumbu x dengan xtick().
\end{eulercomment}
\begin{eulerprompt}
>plot2d("sinc(x)",0,2pi,grid=4,<ticks); ...
>ygrid(-2:0.5:2,grid=6); ...
>xgrid([0:2]*pi,<ticks,grid=6);  ...
>xtick([0,pi,2pi],["0","\(\backslash\)pi","2\(\backslash\)pi"],>latex):
\end{eulerprompt}
\eulerimg{17}{images/22305144006-EMT4Plot2D-065.png}
\begin{eulercomment}
Tentu saja, fungsi juga dapat digunakan.
\end{eulercomment}
\begin{eulerprompt}
>function map f(x) ...
\end{eulerprompt}
\begin{eulerudf}
  if x>0 then return x^4
  else return x^2
  endif
  endfunction
\end{eulerudf}
\begin{eulercomment}
Parameter "map" membantu menggunakan fungsi untuk vektor. Untuk\\
plot, hal ini tidak diperlukan. Tetapi untuk mendemonstrasikan bahwa
vektorisasi\\
berguna, kami menambahkan beberapa titik kunci pada plot pada x=-1,
x=0 dan x=1.

Pada plot berikut, kita juga memasukkan beberapa kode LaTeX. Kita
menggunakannya untuk\\
dua label dan sebuah kotak teks. Tentu saja, Anda hanya dapat
menggunakan\\
LaTeX jika Anda telah menginstal LaTeX dengan benar.
\end{eulercomment}
\begin{eulerprompt}
>plot2d("f",-1,1,xl="x",yl="f(x)",grid=6);  ...
>plot2d([-1,0,1],f([-1,0,1]),>points,>add); ...
>label(latex("x^3"),0.72,f(0.72)); ...
>label(latex("x^2"),-0.52,f(-0.52),pos="ll"); ...
>textbox( ...
>  latex("f(x)=\(\backslash\)begin\{cases\} x^3 & x>0 \(\backslash\)\(\backslash\) x^2 & x \(\backslash\)le 0\(\backslash\)end\{cases\}"), ...
>  x=0.7,y=0.2):
\end{eulerprompt}
\eulerimg{17}{images/22305144006-EMT4Plot2D-066.png}
\begin{eulercomment}
\end{eulercomment}
\eulersubheading{Interaksi Pengguna}
\begin{eulercomment}
Ketika memplot fungsi atau ekspresi, parameter \textgreater{}user memungkinkan
pengguna untuk memperbesar dan menggeser plot dengan tombol kursor
atau mouse. Pengguna dapat

- memperbesar dengan + atau -\\
- memindahkan plot dengan tombol kursor\\
- memilih jendela plot dengan mouse\\
- mengatur ulang tampilan dengan spasi\\
- keluar dengan return

Tombol spasi akan mengatur ulang plot ke jendela plot awal.

Ketika memplot data, bendera \textgreater{}user hanya akan menunggu penekanan
tombol.
\end{eulercomment}
\begin{eulerprompt}
>plot2d(\{\{"x^3-a*x",a=1\}\},>user,title="Press any key!"):
\end{eulerprompt}
\eulerimg{17}{images/22305144006-EMT4Plot2D-067.png}
\begin{eulerprompt}
>plot2d("exp(x)*sin(x)",user=true, ...
>  title="+/- or cursor keys (return to exit)"):
\end{eulerprompt}
\eulerimg{17}{images/22305144006-EMT4Plot2D-068.png}
\begin{eulercomment}
Berikut ini menunjukkan cara interaksi pengguna tingkat lanjut (lihat
tutorial tentang pemrograman untuk detailnya).

Fungsi bawaan mousedrag() menunggu peristiwa mouse atau keyboard.
Fungsi ini melaporkan mouse ke bawah, mouse bergerak atau mouse ke
atas, dan penekanan tombol. Fungsi dragpoints() memanfaatkan hal ini,
dan mengizinkan pengguna untuk menyeret titik manapun di dalam plot.

Kita membutuhkan fungsi plot terlebih dahulu. Sebagai contoh, kita
melakukan interpolasi pada 5 titik dengan sebuah polinomial. Fungsi
ini harus memplot ke dalam area plot yang tetap.
\end{eulercomment}
\begin{eulerprompt}
>function plotf(xp,yp,select) ...
\end{eulerprompt}
\begin{eulerudf}
    d=interp(xp,yp);
    plot2d("interpval(xp,d,x)";d,xp,r=2);
    plot2d(xp,yp,>points,>add);
    if select>0 then
      plot2d(xp[select],yp[select],color=red,>points,>add);
    endif;
    title("Drag one point, or press space or return!");
  endfunction
\end{eulerudf}
\begin{eulercomment}
Perhatikan parameter titik koma pada plot2d (d dan xp), yang
diteruskan ke evaluasi fungsi interp(). Tanpa ini, kita harus menulis
fungsi plotinterp() terlebih dahulu, untuk mengakses nilai secara
global.

Sekarang kita menghasilkan beberapa nilai acak, dan membiarkan
pengguna menyeret titik-titiknya.
\end{eulercomment}
\begin{eulerprompt}
>t=-1:0.5:1; dragpoints("plotf",t,random(size(t))-0.5):
\end{eulerprompt}
\eulerimg{27}{images/22305144006-EMT4Plot2D-069.png}
\begin{eulercomment}
Ada juga fungsi yang memplot fungsi lain tergantung pada vektor
parameter, dan memungkinkan pengguna menyesuaikan parameter ini.

Pertama, kita memerlukan fungsi plot.
\end{eulercomment}
\begin{eulerprompt}
>function plotf([a,b]) := plot2d("exp(a*x)*cos(2pi*b*x)",0,2pi;a,b);
\end{eulerprompt}
\begin{eulercomment}
Kemudian kita membutuhkan nama untuk parameter, nilai awal dan matriks
rentang nx2, dan secara opsional, sebuah garis judul.\\
Terdapat slider interaktif, yang dapat mengatur nilai oleh pengguna.
Fungsi dragvalues() menyediakan ini.
\end{eulercomment}
\begin{eulerprompt}
>dragvalues("plotf",["a","b"],[-1,2],[[-2,2];[1,10]], ...
>  heading="Drag these values:",hcolor=black):
\end{eulerprompt}
\eulerimg{17}{images/22305144006-EMT4Plot2D-070.png}
\begin{eulercomment}
Anda dapat membatasi nilai yang diseret menjadi bilangan bulat.
Sebagai contoh, kita menulis fungsi plot, yang memplot polinomial
Taylor dengan derajat n ke fungsi kosinus.
\end{eulercomment}
\begin{eulerprompt}
>function plotf(n) ...
\end{eulerprompt}
\begin{eulerudf}
  plot2d("cos(x)",0,2pi,>square,grid=6);
  plot2d(&"taylor(cos(x),x,0,@n)",color=blue,>add);
  textbox("Taylor polynomial of degree "+n,0.1,0.02,style="t",>left);
  endfunction
\end{eulerudf}
\begin{eulercomment}
Sekarang kita membiarkan derajat n bervariasi dari 0 sampai 20 dalam
20 stop. Hasil dari dragvalues() digunakan untuk memplot sketsa dengan
n ini, dan untuk menyisipkan plot ke dalam buku catatan.
\end{eulercomment}
\begin{eulerprompt}
>nd=dragvalues("plotf","degree",2,[0,20],20,y=0.8, ...
>   heading="Drag the value:"); ...
>plotf(nd):
\end{eulerprompt}
\eulerimg{17}{images/22305144006-EMT4Plot2D-071.png}
\begin{eulercomment}
Berikut ini adalah peragaan sederhana dari fungsi ini. Pengguna dapat
menggambar di atas jendela plot, meninggalkan jejak titik.
\end{eulercomment}
\begin{eulerprompt}
>function dragtest ...
\end{eulerprompt}
\begin{eulerudf}
    plot2d(none,r=1,title="Drag with the mouse, or press any key!");
    start=0;
    repeat
      \{flag,m,time\}=mousedrag();
      if flag==0 then return; endif;
      if flag==2 then
        hold on; mark(m[1],m[2]); hold off;
      endif;
    end
  endfunction
\end{eulerudf}
\begin{eulerprompt}
>dragtest // lihat hasilnya dan cobalah lakukan!
\end{eulerprompt}
\eulersubheading{Gaya Plot 2D}
\begin{eulercomment}
Secara default, EMT menghitung tanda sumbu otomatis dan menambahkan
label pada setiap tanda. Hal ini dapat diubah dengan parameter grid.
Gaya default sumbu dan label dapat dimodifikasi. Selain itu, label dan
judul dapat ditambahkan secara manual. Untuk mengatur ulang ke gaya
default, gunakan reset().
\end{eulercomment}
\begin{eulerprompt}
>aspect();
>figure(3,4); ...
> figure(1); plot2d("x^3-x",grid=0); ... // no grid, frame or axis
> figure(2); plot2d("x^3-x",grid=1); ... // x-y-axis
> figure(3); plot2d("x^3-x",grid=2); ... // default ticks
> figure(4); plot2d("x^3-x",grid=3); ... // x-y- axis with labels inside
> figure(5); plot2d("x^3-x",grid=4); ... // no ticks, only labels
> figure(6); plot2d("x^3-x",grid=5); ... // default, but no margin
> figure(7); plot2d("x^3-x",grid=6); ... // axes only
> figure(8); plot2d("x^3-x",grid=7); ... // axes only, ticks at axis
> figure(9); plot2d("x^3-x",grid=8); ... // axes only, finer ticks at axis
> figure(10); plot2d("x^3-x",grid=9); ... // default, small ticks inside
> figure(11); plot2d("x^3-x",grid=10); ...// no ticks, axes only
> figure(0):
\end{eulerprompt}
\eulerimg{27}{images/22305144006-EMT4Plot2D-072.png}
\begin{eulercomment}
Parameter \textless{}frame mematikan bingkai, dan framecolor=blue menetapkan
bingkai ke warna biru.

Jika Anda menginginkan tanda centang Anda sendiri, Anda dapat
menggunakan style=0, dan menambahkan semuanya nanti.
\end{eulercomment}
\begin{eulerprompt}
>aspect(1.5); 
>plot2d("x^3-x",grid=0); // plot
>frame; xgrid([-1,0,1]); ygrid(0): // add frame and grid
\end{eulerprompt}
\eulerimg{17}{images/22305144006-EMT4Plot2D-073.png}
\begin{eulercomment}
Untuk judul plot dan label sumbu, lihat contoh berikut.
\end{eulercomment}
\begin{eulerprompt}
>plot2d("exp(x)",-1,1);
>textcolor(black); // set the text color to black
>title(latex("y=e^x")); // title above the plot
>xlabel(latex("x")); // "x" for x-axis
>ylabel(latex("y"),>vertical); // vertical "y" for y-axis
>label(latex("(0,1)"),0,1,color=blue): // label a point
\end{eulerprompt}
\eulerimg{17}{images/22305144006-EMT4Plot2D-074.png}
\begin{eulercomment}
Sumbu dapat digambar secara terpisah dengan sumbu x() dan sumbu y().
\end{eulercomment}
\begin{eulerprompt}
>+plot2d("x^3-x",<grid,<frame);
>xaxis(0,xx=-2:1,style="->"); yaxis(0,yy=-5:5,style="->"):
\end{eulerprompt}
\eulerimg{17}{images/22305144006-EMT4Plot2D-075.png}
\begin{eulercomment}
Teks pada plot dapat diatur dengan label(). Pada contoh berikut ini,
"lc" berarti lower center. Ini mengatur posisi label relatif terhadap
koordinat plot.
\end{eulercomment}
\begin{eulerprompt}
>function f(x) &= x^3-x
\end{eulerprompt}
\begin{euleroutput}
  
                                   3
                                  x  - x
  
\end{euleroutput}
\begin{eulerprompt}
>plot2d(f,-1,1,>square);
>x0=fmin(f,0,1); // compute point of minimum
>label("Rel. Min.",x0,f(x0),pos="lc"): // add a label there
\end{eulerprompt}
\eulerimg{17}{images/22305144006-EMT4Plot2D-076.png}
\begin{eulercomment}
Terdapat juga kotak teks.
\end{eulercomment}
\begin{eulerprompt}
>plot2d(&f(x),-1,1,-2,2); // function
>plot2d(&diff(f(x),x),>add,style="--",color=red); // derivative
>labelbox(["f","f'"],["-","--"],[black,red]): // label box
\end{eulerprompt}
\eulerimg{17}{images/22305144006-EMT4Plot2D-077.png}
\begin{eulerprompt}
>plot2d(["exp(x)","1+x"],color=[black,blue],style=["-","-.-"]):
\end{eulerprompt}
\eulerimg{17}{images/22305144006-EMT4Plot2D-078.png}
\begin{eulerprompt}
>gridstyle("->",color=gray,textcolor=gray,framecolor=gray);  ...
> plot2d("x^3-x",grid=1);   ...
> settitle("y=x^3-x",color=black); ...
> label("x",2,0,pos="bc",color=gray);  ...
> label("y",0,6,pos="cl",color=gray); ...
> reset():
\end{eulerprompt}
\eulerimg{27}{images/22305144006-EMT4Plot2D-079.png}
\begin{eulercomment}
Untuk kontrol yang lebih besar lagi, sumbu x dan sumbu y dapat
dilakukan secara manual.

Perintah fullwindow() akan memperluas jendela plot karena kita tidak
lagi membutuhkan tempat untuk label di luar jendela plot. Gunakan
shrinkwindow() atau reset() untuk mengatur ulang ke default.
\end{eulercomment}
\begin{eulerprompt}
>fullwindow; ...
> gridstyle(color=darkgray,textcolor=darkgray); ...
> plot2d(["2^x","1","2^(-x)"],a=-2,b=2,c=0,d=4,<grid,color=4:6,<frame); ...
> xaxis(0,-2:1,style="->"); xaxis(0,2,"x",<axis); ...
> yaxis(0,4,"y",style="->"); ...
> yaxis(-2,1:4,>left); ...
> yaxis(2,2^(-2:2),style=".",<left); ...
> labelbox(["2^x","1","2^-x"],colors=4:6,x=0.8,y=0.2); ...
> reset:
\end{eulerprompt}
\eulerimg{27}{images/22305144006-EMT4Plot2D-080.png}
\begin{eulercomment}
Berikut ini adalah contoh lain, di mana string Unicode digunakan dan
sumbu di luar area plot.
\end{eulercomment}
\begin{eulerprompt}
>aspect(1.5); 
>plot2d(["sin(x)","cos(x)"],0,2pi,color=[red,green],<grid,<frame); ...
> xaxis(-1.1,(0:2)*pi,xt=["0",u"&pi;",u"2&pi;"],style="-",>ticks,>zero);  ...
> xgrid((0:0.5:2)*pi,<ticks); ...
> yaxis(-0.1*pi,-1:0.2:1,style="-",>zero,>grid); ...
> labelbox(["sin","cos"],colors=[red,green],x=0.5,y=0.2,>left); ...
> xlabel(u"&phi;"); ylabel(u"f(&phi;)"):
\end{eulerprompt}
\eulerimg{17}{images/22305144006-EMT4Plot2D-081.png}
\eulerheading{Memplot Data 2D}
\begin{eulercomment}
Jika x dan y adalah vektor data, data ini akan digunakan sebagai
koordinat x dan y dari sebuah kurva. Dalam hal ini, a, b, c, dan d,
atau radius r dapat ditentukan, atau jendela plot akan menyesuaikan
secara otomatis dengan data. Sebagai alternatif, \textgreater{}square dapat diatur
untuk mempertahankan rasio aspek persegi.

Memplot ekspresi hanyalah singkatan untuk plot data. Untuk plot data,
Anda memerlukan satu atau beberapa baris nilai x, dan satu atau
beberapa baris nilai y. Dari rentang dan nilai x, fungsi plot2d akan
menghitung data untuk diplot, secara default dengan evaluasi adaptif
dari fungsi tersebut. Untuk plot titik, gunakan "\textgreater{}points", untuk garis
dan titik campuran gunakan "\textgreater{}addpoints".

Namun Anda dapat memasukkan data secara langsung.

- Gunakan vektor baris untuk x dan y untuk satu fungsi.\\
- Matriks untuk x dan y diplot baris demi baris.

Berikut adalah contoh dengan satu baris untuk x dan y.
\end{eulercomment}
\begin{eulerprompt}
>x=-10:0.1:10; y=exp(-x^2)*x; plot2d(x,y):
\end{eulerprompt}
\eulerimg{17}{images/22305144006-EMT4Plot2D-082.png}
\begin{eulercomment}
Data juga dapat diplot sebagai titik. Gunakan poin=true untuk ini.
Plot ini bekerja seperti poligon, namun hanya menggambar
sudut-sudutnya saja.

- style = "...": Pilih dari "[]", "\textless{}\textgreater{}", "o", ".", "..", "+", "*", "[]
#", "\textless{}\textgreater{}#", "o#", "..#", "#", "\textbar{}".

Untuk memplot kumpulan titik, gunakan \textgreater{}titik. Jika warna adalah sebuah
vektor warna, setiap titik\\
mendapatkan warna yang berbeda. Untuk sebuah matriks koordinat dan
vektor kolom, warna berlaku pada baris-baris matriks.\\
Parameter \textgreater{}addpoints menambahkan titik-titik pada segmen garis untuk
plot data.
\end{eulercomment}
\begin{eulerprompt}
>xdata=[1,1.5,2.5,3,4]; ydata=[3,3.1,2.8,2.9,2.7]; // data
>plot2d(xdata,ydata,a=0.5,b=4.5,c=2.5,d=3.5,style="."); // lines
>plot2d(xdata,ydata,>points,>add,style="o"): // add points
\end{eulerprompt}
\eulerimg{17}{images/22305144006-EMT4Plot2D-083.png}
\begin{eulerprompt}
>p=polyfit(xdata,ydata,1); // get regression line
>plot2d("polyval(p,x)",>add,color=red): // add plot of line
\end{eulerprompt}
\eulerimg{17}{images/22305144006-EMT4Plot2D-084.png}
\begin{eulercomment}
Plot data benar-benar berupa poligon. Kita juga dapat memplot kurva
atau kurva terisi.

- filled=true mengisi plot.\\
- style = "...": Pilih dari "#", "/", "\textbackslash{}", "\textbackslash{}/".\\
- fillcolor: Lihat di atas untuk warna yang tersedia.

Warna isian ditentukan oleh argumen "fillcolor", dan pada pilihan
\textless{}outline mencegah menggambar batas untuk semua gaya kecuali gaya
default.
\end{eulercomment}
\begin{eulerprompt}
>t=linspace(0,2pi,1000); // parameter for curve
>x=sin(t)*exp(t/pi); y=cos(t)*exp(t/pi); // x(t) and y(t)
>figure(1,2); aspect(16/9)
>figure(1); plot2d(x,y,r=10); // plot curve
>figure(2); plot2d(x,y,r=10,>filled,style="/",fillcolor=red); // fill curve
>figure(0):
\end{eulerprompt}
\eulerimg{14}{images/22305144006-EMT4Plot2D-085.png}
\begin{eulercomment}
Pada contoh berikut ini, kami memplot elips terisi dan dua segi enam
terisi menggunakan kurva tertutup dengan 6 titik dengan gaya isian
yang berbeda.
\end{eulercomment}
\begin{eulerprompt}
>x=linspace(0,2pi,1000); plot2d(sin(x),cos(x)*0.5,r=1,>filled,style="/"):
\end{eulerprompt}
\eulerimg{14}{images/22305144006-EMT4Plot2D-086.png}
\begin{eulerprompt}
>t=linspace(0,2pi,6); ...
>plot2d(cos(t),sin(t),>filled,style="/",fillcolor=red,r=1.2):
\end{eulerprompt}
\eulerimg{14}{images/22305144006-EMT4Plot2D-087.png}
\begin{eulerprompt}
>t=linspace(0,2pi,6); plot2d(cos(t),sin(t),>filled,style="#"):
\end{eulerprompt}
\eulerimg{14}{images/22305144006-EMT4Plot2D-088.png}
\begin{eulercomment}
Contoh lainnya adalah septagon, yang kita buat dengan 7 titik pada
lingkaran satuan.
\end{eulercomment}
\begin{eulerprompt}
>t=linspace(0,2pi,7);  ...
> plot2d(cos(t),sin(t),r=1,>filled,style="/",fillcolor=red):
\end{eulerprompt}
\eulerimg{14}{images/22305144006-EMT4Plot2D-089.png}
\begin{eulercomment}
Berikut ini adalah himpunan nilai maksimal dari empat kondisi linier
yang kurang dari atau sama dengan 3. Ini adalah A[k].v\textless{}=3 untuk semua
barisan A. Untuk mendapatkan sudut-sudut yang bagus, kita menggunakan
n yang relatif besar.
\end{eulercomment}
\begin{eulerprompt}
>A=[2,1;1,2;-1,0;0,-1];
>function f(x,y) := max([x,y].A');
>plot2d("f",r=4,level=[0;3],color=green,n=111):
\end{eulerprompt}
\eulerimg{14}{images/22305144006-EMT4Plot2D-090.png}
\begin{eulercomment}
Poin utama dari bahasa matriks adalah bahwa bahasa ini memungkinkan
untuk menghasilkan tabel fungsi dengan mudah.
\end{eulercomment}
\begin{eulerprompt}
>t=linspace(0,2pi,1000); x=cos(3*t); y=sin(4*t);
\end{eulerprompt}
\begin{eulercomment}
Kita sekarang memiliki vektor nilai x dan y. plot2d() dapat memplot
nilai-nilai ini\\
sebagai sebuah kurva yang menghubungkan titik-titik. Plot dapat diisi.
Dalam kasus ini\\
ini memberikan hasil yang bagus karena aturan penggulungan, yang
digunakan untuk\\
pengisian.
\end{eulercomment}
\begin{eulerprompt}
>plot2d(x,y,<grid,<frame,>filled):
\end{eulerprompt}
\eulerimg{14}{images/22305144006-EMT4Plot2D-091.png}
\begin{eulercomment}
Vektor interval diplot terhadap nilai x sebagai wilayah yang terisi\\
antara nilai bawah dan atas interval.

Hal ini dapat berguna untuk memplot kesalahan perhitungan. Tapi itu
bisa\\
juga dapat digunakan untuk memplot kesalahan statistik.
\end{eulercomment}
\begin{eulerprompt}
>t=0:0.1:1; ...
> plot2d(t,interval(t-random(size(t)),t+random(size(t))),style="|");  ...
> plot2d(t,t,add=true):
\end{eulerprompt}
\eulerimg{14}{images/22305144006-EMT4Plot2D-092.png}
\begin{eulercomment}
Jika x adalah vektor yang diurutkan, dan y adalah vektor interval,
maka plot2d akan memplot rentang interval yang terisi pada bidang,
gaya isian sama dengan gaya poligon.
\end{eulercomment}
\begin{eulerprompt}
>t=-1:0.01:1; x=~t-0.01,t+0.01~; y=x^3-x;
>plot2d(t,y):
\end{eulerprompt}
\eulerimg{14}{images/22305144006-EMT4Plot2D-093.png}
\begin{eulercomment}
Dimungkinkan untuk mengisi wilayah nilai untuk fungsi tertentu. Untuk\\
ini, level harus berupa matriks 2xn. Baris pertama adalah batas bawah\\
dan baris kedua berisi batas atas.
\end{eulercomment}
\begin{eulerprompt}
>expr := "2*x^2+x*y+3*y^4+y"; // define an expression f(x,y)
>plot2d(expr,level=[0;1],style="-",color=blue): // 0 <= f(x,y) <= 1
\end{eulerprompt}
\eulerimg{14}{images/22305144006-EMT4Plot2D-094.png}
\begin{eulercomment}
Kita juga dapat mengisi rentang nilai seperti

\end{eulercomment}
\begin{eulerformula}
\[
-1 \le (x^2+y^2)^2-x^2+y^2 \le 0.
\]
\end{eulerformula}
\begin{eulercomment}
\end{eulercomment}
\begin{eulerprompt}
>plot2d("(x^2+y^2)^2-x^2+y^2",r=1.2,level=[-1;0],style="/"):
\end{eulerprompt}
\eulerimg{14}{images/22305144006-EMT4Plot2D-096.png}
\begin{eulerprompt}
>plot2d("cos(x)","sin(x)^3",xmin=0,xmax=2pi,>filled,style="/"):
\end{eulerprompt}
\eulerimg{14}{images/22305144006-EMT4Plot2D-097.png}
\eulerheading{Grafik Fungsi Parametrik}
\begin{eulercomment}
Nilai x tidak perlu diurutkan. (x,y) hanya menggambarkan sebuah kurva.
Jika x diurutkan, kurva tersebut adalah grafik fungsi.

Pada contoh berikut, kita memplot spiral

\end{eulercomment}
\begin{eulerformula}
\[
\gamma(t) = t \cdot (\cos(2\pi t),\sin(2\pi t))
\]
\end{eulerformula}
\begin{eulercomment}
Kita mungkin perlu menggunakan sangat banyak titik untuk tampilan yang
halus atau fungsi adaptive() untuk mengevaluasi ekspresi (lihat fungsi
adaptive() untuk lebih jelasnya).
\end{eulercomment}
\begin{eulerprompt}
>t=linspace(0,1,1000); ...
>plot2d(t*cos(2*pi*t),t*sin(2*pi*t),r=1):
\end{eulerprompt}
\eulerimg{14}{images/22305144006-EMT4Plot2D-099.png}
\begin{eulercomment}
Sebagai alternatif, Anda dapat menggunakan dua ekspresi untuk kurva.
Berikut ini memplot kurva yang sama seperti di atas.
\end{eulercomment}
\begin{eulerprompt}
>plot2d("x*cos(2*pi*x)","x*sin(2*pi*x)",xmin=0,xmax=1,r=1):
\end{eulerprompt}
\eulerimg{14}{images/22305144006-EMT4Plot2D-100.png}
\begin{eulerprompt}
>t=linspace(0,1,1000); r=exp(-t); x=r*cos(2pi*t); y=r*sin(2pi*t);
>plot2d(x,y,r=1):
\end{eulerprompt}
\eulerimg{14}{images/22305144006-EMT4Plot2D-101.png}
\begin{eulercomment}
Pada contoh berikut ini, kami memplot kurva

\end{eulercomment}
\begin{eulerformula}
\[
\gamma(t) = (r(t) \cos(t), r(t) \sin(t))
\]
\end{eulerformula}
\begin{eulercomment}
dengan

\end{eulercomment}
\begin{eulerformula}
\[
r(t) = 1 + \dfrac{\sin(3t)}{2}.
\]
\end{eulerformula}
\begin{eulerprompt}
>t=linspace(0,2pi,1000); r=1+sin(3*t)/2; x=r*cos(t); y=r*sin(t); ...
>plot2d(x,y,>filled,fillcolor=red,style="/",r=1.5):
\end{eulerprompt}
\eulerimg{14}{images/22305144006-EMT4Plot2D-104.png}
\eulerheading{Menggambar Grafik Bilangan Kompleks}
\begin{eulercomment}
Sebuah deretan bilangan kompleks juga dapat diplot. Kemudian
titik-titik kisi akan dihubungkan. Jika sejumlah garis kisi ditentukan
(atau vektor 1x2 garis kisi) pada argumen cgrid, hanya garis-garis
kisi tersebut yang akan terlihat.

Matriks bilangan kompleks akan secara otomatis diplot sebagai sebuah
grid pada bidang kompleks.

Pada contoh berikut, kita memplot gambar lingkaran satuan di bawah
fungsi eksponensial. Parameter cgrid menyembunyikan beberapa kurva
grid.
\end{eulercomment}
\begin{eulerprompt}
>aspect(); r=linspace(0,1,50); a=linspace(0,2pi,80)'; z=r*exp(I*a);...
>plot2d(z,a=-1.25,b=1.25,c=-1.25,d=1.25,cgrid=10):
\end{eulerprompt}
\eulerimg{27}{images/22305144006-EMT4Plot2D-105.png}
\begin{eulerprompt}
>aspect(1.25); r=linspace(0,1,50); a=linspace(0,2pi,200)'; z=r*exp(I*a);
>plot2d(exp(z),cgrid=[40,10]):
\end{eulerprompt}
\eulerimg{21}{images/22305144006-EMT4Plot2D-106.png}
\begin{eulerprompt}
>r=linspace(0,1,10); a=linspace(0,2pi,40)'; z=r*exp(I*a);
>plot2d(exp(z),>points,>add):
\end{eulerprompt}
\eulerimg{21}{images/22305144006-EMT4Plot2D-107.png}
\begin{eulercomment}
Vektor bilangan kompleks secara otomatis diplot sebagai kurva pada
bidang kompleks dengan bagian nyata dan bagian imajiner.

Pada contoh, kami memplot lingkaran satuan dengan

\end{eulercomment}
\begin{eulerformula}
\[
\gamma(t) = e^{it}
\]
\end{eulerformula}
\begin{eulerprompt}
>t=linspace(0,2pi,1000); ...
>plot2d(exp(I*t)+exp(4*I*t),r=2):
\end{eulerprompt}
\eulerimg{21}{images/22305144006-EMT4Plot2D-109.png}
\eulerheading{Plot Statistik}
\begin{eulercomment}
Terdapat banyak fungsi yang dikhususkan untuk plot statistik. Salah
satu plot yang sering digunakan adalah plot kolom.

Jumlah kumulatif dari nilai berdistribusi normal 0-1 menghasilkan
jalan acak.
\end{eulercomment}
\begin{eulerprompt}
>plot2d(cumsum(randnormal(1,1000))):
\end{eulerprompt}
\eulerimg{21}{images/22305144006-EMT4Plot2D-110.png}
\begin{eulercomment}
Dengan menggunakan dua baris, ini menunjukkan jalan kaki dalam dua
dimensi.
\end{eulercomment}
\begin{eulerprompt}
>X=cumsum(randnormal(2,1000)); plot2d(X[1],X[2]):
\end{eulerprompt}
\eulerimg{21}{images/22305144006-EMT4Plot2D-111.png}
\begin{eulerprompt}
>columnsplot(cumsum(random(10)),style="/",color=blue):
\end{eulerprompt}
\eulerimg{21}{images/22305144006-EMT4Plot2D-112.png}
\begin{eulercomment}
Ini juga dapat menampilkan string sebagai label.
\end{eulercomment}
\begin{eulerprompt}
>months=["Jan","Feb","Mar","Apr","May","Jun", ...
>  "Jul","Aug","Sep","Oct","Nov","Dec"];
>values=[10,12,12,18,22,28,30,26,22,18,12,8];
>columnsplot(values,lab=months,color=red,style="-");
>title("Temperature"):
\end{eulerprompt}
\eulerimg{21}{images/22305144006-EMT4Plot2D-113.png}
\begin{eulerprompt}
>k=0:10;
>plot2d(k,bin(10,k),>bar):
\end{eulerprompt}
\eulerimg{21}{images/22305144006-EMT4Plot2D-114.png}
\begin{eulerprompt}
>plot2d(k,bin(10,k)); plot2d(k,bin(10,k),>points,>add):
\end{eulerprompt}
\eulerimg{21}{images/22305144006-EMT4Plot2D-115.png}
\begin{eulerprompt}
>plot2d(normal(1000),normal(1000),>points,grid=6,style=".."):
\end{eulerprompt}
\eulerimg{21}{images/22305144006-EMT4Plot2D-116.png}
\begin{eulerprompt}
>plot2d(normal(1,1000),>distribution,style="O"):
\end{eulerprompt}
\eulerimg{21}{images/22305144006-EMT4Plot2D-117.png}
\begin{eulerprompt}
>plot2d("qnormal",0,5;2.5,0.5,>filled):
\end{eulerprompt}
\eulerimg{21}{images/22305144006-EMT4Plot2D-118.png}
\begin{eulercomment}
Untuk memplot distribusi statistik eksperimental, Anda dapat
menggunakan distribution=n dengan plot2d.
\end{eulercomment}
\begin{eulerprompt}
>w=randexponential(1,1000); // exponential distribution
>plot2d(w,>distribution): // or distribution=n with n intervals
\end{eulerprompt}
\eulerimg{21}{images/22305144006-EMT4Plot2D-119.png}
\begin{eulercomment}
Atau Anda dapat menghitung distribusi dari data dan memplot hasilnya
dengan \textgreater{}bar di plot3d, atau dengan plot kolom.
\end{eulercomment}
\begin{eulerprompt}
>w=normal(1000); // 0-1-normal distribution
>\{x,y\}=histo(w,10,v=[-6,-4,-2,-1,0,1,2,4,6]); // interval bounds v
>plot2d(x,y,>bar):
\end{eulerprompt}
\eulerimg{21}{images/22305144006-EMT4Plot2D-120.png}
\begin{eulercomment}
Fungsi statplot() menetapkan gaya dengan string sederhana.
\end{eulercomment}
\begin{eulerprompt}
>statplot(1:10,cumsum(random(10)),"b"):
\end{eulerprompt}
\eulerimg{21}{images/22305144006-EMT4Plot2D-121.png}
\begin{eulerprompt}
>n=10; i=0:n; ...
>plot2d(i,bin(n,i)/2^n,a=0,b=10,c=0,d=0.3); ...
>plot2d(i,bin(n,i)/2^n,points=true,style="ow",add=true,color=blue):
\end{eulerprompt}
\eulerimg{21}{images/22305144006-EMT4Plot2D-122.png}
\begin{eulercomment}
Selain itu, data dapat diplot sebagai batang. Dalam hal ini, x harus
diurutkan dan satu elemen lebih panjang dari y. Batang akan memanjang
dari x[i] ke x[i+1] dengan nilai y[i]. Jika x memiliki ukuran yang
sama dengan y, maka x akan diperpanjang satu elemen dengan jarak
terakhir.

Gaya isian dapat digunakan seperti di atas.
\end{eulercomment}
\begin{eulerprompt}
>n=10; k=bin(n,0:n); ...
>plot2d(-0.5:n+0.5,k,bar=true,fillcolor=lightgray):
\end{eulerprompt}
\eulerimg{21}{images/22305144006-EMT4Plot2D-123.png}
\begin{eulercomment}
Data untuk plot batang (batang = 1) dan histogram (histogram = 1)
dapat diberikan secara eksplisit dalam xv dan yv, atau dapat dihitung
dari distribusi empiris dalam xv dengan \textgreater{}distribution (atau distribusi
= n). Histogram dari nilai xv akan dihitung secara otomatis dengan
\textgreater{}histogram. Jika \textgreater{}even ditentukan, nilai xv akan dihitung dalam
interval bilangan bulat.
\end{eulercomment}
\begin{eulerprompt}
>plot2d(normal(10000),distribution=50):
\end{eulerprompt}
\eulerimg{21}{images/22305144006-EMT4Plot2D-124.png}
\begin{eulerprompt}
>k=0:10; m=bin(10,k); x=(0:11)-0.5; plot2d(x,m,>bar):
\end{eulerprompt}
\eulerimg{21}{images/22305144006-EMT4Plot2D-125.png}
\begin{eulerprompt}
>columnsplot(m,k):
\end{eulerprompt}
\eulerimg{21}{images/22305144006-EMT4Plot2D-126.png}
\begin{eulerprompt}
>plot2d(random(600)*6,histogram=6):
\end{eulerprompt}
\eulerimg{21}{images/22305144006-EMT4Plot2D-127.png}
\begin{eulercomment}
Untuk distribusi, ada parameter distribution=n, yang menghitung nilai
secara otomatis dan mencetak distribusi relatif dengan n sub-interval.
\end{eulercomment}
\begin{eulerprompt}
>plot2d(normal(1,1000),distribution=10,style="\(\backslash\)/"):
\end{eulerprompt}
\eulerimg{21}{images/22305144006-EMT4Plot2D-128.png}
\begin{eulercomment}
Dengan parameter even=true, ini akan menggunakan interval bilangan
bulat.
\end{eulercomment}
\begin{eulerprompt}
>plot2d(intrandom(1,1000,10),distribution=10,even=true):
\end{eulerprompt}
\eulerimg{21}{images/22305144006-EMT4Plot2D-129.png}
\begin{eulercomment}
Perhatikan bahwa ada banyak plot statistik yang mungkin berguna.
Lihatlah tutorial tentang statistik.
\end{eulercomment}
\begin{eulerprompt}
>columnsplot(getmultiplicities(1:6,intrandom(1,6000,6))):
\end{eulerprompt}
\eulerimg{21}{images/22305144006-EMT4Plot2D-130.png}
\begin{eulerprompt}
>plot2d(normal(1,1000),>distribution); ...
>  plot2d("qnormal(x)",color=red,thickness=2,>add):
\end{eulerprompt}
\eulerimg{21}{images/22305144006-EMT4Plot2D-131.png}
\begin{eulercomment}
Ada juga banyak plot khusus untuk statistik. Boxplot menunjukkan
kuartil dari distribusi ini dan banyak pencilan. Menurut definisi,
pencilan dalam boxplot adalah data yang melebihi 1,5 kali kisaran 50\%
tengah plot.
\end{eulercomment}
\begin{eulerprompt}
>M=normal(5,1000); boxplot(quartiles(M)):
\end{eulerprompt}
\eulerimg{21}{images/22305144006-EMT4Plot2D-132.png}
\eulerheading{Fungsi Implisit}
\begin{eulercomment}
Plot implisit menunjukkan garis level yang menyelesaikan f(x,y)=level,
di mana "level" dapat berupa nilai tunggal atau vektor nilai. Jika
level = "auto", akan ada nc garis level, yang akan menyebar di antara
minimum dan maksimum fungsi secara merata. Warna yang lebih gelap atau
lebih terang dapat ditambahkan dengan \textgreater{}hue untuk mengindikasikan nilai
fungsi. Untuk fungsi implisit, xv haruslah sebuah fungsi atau ekspresi
dari parameter x dan y, atau, sebagai alternatif, xv dapat berupa
matriks nilai.

Euler dapat menandai garis level

\end{eulercomment}
\begin{eulerformula}
\[
f(x,y) = c
\]
\end{eulerformula}
\begin{eulercomment}
dari fungsi apa pun.

Untuk menggambar himpunan f(x,y) = c untuk satu atau lebih konstanta
c, Anda bisa menggunakan plot2d() dengan plot implisitnya pada bidang.
Parameter untuk c adalah level = c, di mana c dapat berupa vektor
garis level. Sebagai tambahan, sebuah skema warna dapat digambar pada
latar belakang untuk mengindikasikan nilai fungsi untuk setiap titik
pada plot. Parameter "n" menentukan kehalusan plot.
\end{eulercomment}
\begin{eulerprompt}
>aspect(1.5); 
>plot2d("x^2+y^2-x*y-x",r=1.5,level=0,contourcolor=red):
\end{eulerprompt}
\eulerimg{27}{images/22305144006-EMT4Plot2D-134.png}
\begin{eulerprompt}
>expr := "2*x^2+x*y+3*y^4+y"; // define an expression f(x,y)
>plot2d(expr,level=0): // Solutions of f(x,y)=0
\end{eulerprompt}
\eulerimg{17}{images/22305144006-EMT4Plot2D-135.png}
\begin{eulerprompt}
>plot2d(expr,level=0:0.5:20,>hue,contourcolor=white,n=200): // nice
\end{eulerprompt}
\eulerimg{17}{images/22305144006-EMT4Plot2D-136.png}
\begin{eulerprompt}
>plot2d(expr,level=0:0.5:20,>hue,>spectral,n=200,grid=4): // nicer
\end{eulerprompt}
\eulerimg{17}{images/22305144006-EMT4Plot2D-137.png}
\begin{eulercomment}
Hal ini juga berlaku untuk plot data. Tetapi Anda harus menentukan
rentang\\
untuk label sumbu.
\end{eulercomment}
\begin{eulerprompt}
>x=-2:0.05:1; y=x'; z=expr(x,y);
>plot2d(z,level=0,a=-1,b=2,c=-2,d=1,>hue):
\end{eulerprompt}
\eulerimg{17}{images/22305144006-EMT4Plot2D-138.png}
\begin{eulerprompt}
>plot2d("x^3-y^2",>contour,>hue,>spectral):
\end{eulerprompt}
\eulerimg{17}{images/22305144006-EMT4Plot2D-139.png}
\begin{eulerprompt}
>plot2d("x^3-y^2",level=0,contourwidth=3,>add,contourcolor=red):
\end{eulerprompt}
\eulerimg{17}{images/22305144006-EMT4Plot2D-140.png}
\begin{eulerprompt}
>z=z+normal(size(z))*0.2;
>plot2d(z,level=0.5,a=-1,b=2,c=-2,d=1):
\end{eulerprompt}
\eulerimg{17}{images/22305144006-EMT4Plot2D-141.png}
\begin{eulerprompt}
>plot2d(expr,level=[0:0.2:5;0.05:0.2:5.05],color=lightgray):
\end{eulerprompt}
\eulerimg{17}{images/22305144006-EMT4Plot2D-142.png}
\begin{eulerprompt}
>plot2d("x^2+y^3+x*y",level=1,r=4,n=100):
\end{eulerprompt}
\eulerimg{17}{images/22305144006-EMT4Plot2D-143.png}
\begin{eulerprompt}
>plot2d("x^2+2*y^2-x*y",level=0:0.1:10,n=100,contourcolor=white,>hue):
\end{eulerprompt}
\eulerimg{17}{images/22305144006-EMT4Plot2D-144.png}
\begin{eulercomment}
Dimungkinkan juga untuk mengisi set

\end{eulercomment}
\begin{eulerformula}
\[
a \le f(x,y) \le b
\]
\end{eulerformula}
\begin{eulercomment}
dengan rentang level.

Dimungkinkan untuk mengisi wilayah nilai untuk fungsi tertentu. Untuk
ini, level harus berupa matriks 2xn. Baris pertama adalah batas bawah
dan baris kedua berisi batas atas.
\end{eulercomment}
\begin{eulerprompt}
>plot2d(expr,level=[0;1],style="-",color=blue): // 0 <= f(x,y) <= 1
\end{eulerprompt}
\eulerimg{17}{images/22305144006-EMT4Plot2D-146.png}
\begin{eulercomment}
Plot implisit juga dapat menunjukkan rentang level. Maka level harus
berupa matriks 2xn interval level, di mana baris pertama berisi awal
dan baris kedua adalah akhir dari setiap interval. Sebagai alternatif,
vektor baris sederhana dapat digunakan untuk level, dan parameter dl
memperluas nilai level ke interval.
\end{eulercomment}
\begin{eulerprompt}
>plot2d("x^4+y^4",r=1.5,level=[0;1],color=blue,style="/"):
\end{eulerprompt}
\eulerimg{17}{images/22305144006-EMT4Plot2D-147.png}
\begin{eulerprompt}
>plot2d("x^2+y^3+x*y",level=[0,2,4;1,3,5],style="/",r=2,n=100):
\end{eulerprompt}
\eulerimg{17}{images/22305144006-EMT4Plot2D-148.png}
\begin{eulerprompt}
>plot2d("x^2+y^3+x*y",level=-10:20,r=2,style="-",dl=0.1,n=100):
\end{eulerprompt}
\eulerimg{17}{images/22305144006-EMT4Plot2D-149.png}
\begin{eulerprompt}
>plot2d("sin(x)*cos(y)",r=pi,>hue,>levels,n=100):
\end{eulerprompt}
\eulerimg{17}{images/22305144006-EMT4Plot2D-150.png}
\begin{eulercomment}
Anda juga dapat menandai suatu wilayah

\end{eulercomment}
\begin{eulerformula}
\[
a \le f(x,y) \le b.
\]
\end{eulerformula}
\begin{eulercomment}
Hal ini dilakukan dengan menambahkan level dengan dua baris.
\end{eulercomment}
\begin{eulerprompt}
>plot2d("(x^2+y^2-1)^3-x^2*y^3",r=1.3, ...
>  style="#",color=red,<outline, ...
>  level=[-2;0],n=100):
\end{eulerprompt}
\eulerimg{17}{images/22305144006-EMT4Plot2D-152.png}
\begin{eulercomment}
Dimungkinkan untuk menentukan level tertentu. Misalnya, kita dapat
memplot solusi dari persamaan seperti

\end{eulercomment}
\begin{eulerformula}
\[
x^3-xy+x^2y^2=6
\]
\end{eulerformula}
\begin{eulerprompt}
>plot2d("x^3-x*y+x^2*y^2",r=6,level=1,n=100):
\end{eulerprompt}
\eulerimg{17}{images/22305144006-EMT4Plot2D-154.png}
\begin{eulerprompt}
>function starplot1 (v, style="/", color=green, lab=none) ...
\end{eulerprompt}
\begin{eulerudf}
    if !holding() then clg; endif;
    w=window(); window(0,0,1024,1024);
    h=holding(1);
    r=max(abs(v))*1.2;
    setplot(-r,r,-r,r);
    n=cols(v); t=linspace(0,2pi,n);
    v=v|v[1]; c=v*cos(t); s=v*sin(t);
    cl=barcolor(color); st=barstyle(style);
    loop 1 to n
      polygon([0,c[#],c[#+1]],[0,s[#],s[#+1]],1);
      if lab!=none then
        rlab=v[#]+r*0.1;
        \{col,row\}=toscreen(cos(t[#])*rlab,sin(t[#])*rlab);
        ctext(""+lab[#],col,row-textheight()/2);
      endif;
    end;
    barcolor(cl); barstyle(st);
    holding(h);
    window(w);
  endfunction
\end{eulerudf}
\begin{eulercomment}
Tidak ada kisi-kisi atau kutu sumbu di sini. Selain itu, kami
menggunakan jendela penuh untuk plot.

Kami memanggil reset sebelum kami menguji plot ini untuk mengembalikan
default grafis. Hal ini tidak perlu dilakukan, jika Anda yakin bahwa
plot Anda berfungsi.
\end{eulercomment}
\begin{eulerprompt}
>reset; starplot1(normal(1,10)+5,color=red,lab=1:10):
\end{eulerprompt}
\eulerimg{27}{images/22305144006-EMT4Plot2D-155.png}
\begin{eulercomment}
Terkadang, Anda mungkin ingin memplot sesuatu yang tidak dapat
dilakukan oleh plot2d, tetapi hampir.

Pada fungsi berikut ini, kita akan membuat plot impuls logaritmik.
plot2d dapat membuat plot logaritmik, tetapi tidak untuk batang
impuls.
\end{eulercomment}
\begin{eulerprompt}
>function logimpulseplot1 (x,y) ...
\end{eulerprompt}
\begin{eulerudf}
    \{x0,y0\}=makeimpulse(x,log(y)/log(10));
    plot2d(x0,y0,>bar,grid=0);
    h=holding(1);
    frame();
    xgrid(ticks(x));
    p=plot();
    for i=-10 to 10;
      if i<=p[4] and i>=p[3] then
         ygrid(i,yt="10^"+i);
      endif;
    end;
    holding(h);
  endfunction
\end{eulerudf}
\begin{eulercomment}
Mari kita uji dengan nilai yang terdistribusi secara eksponensial.
\end{eulercomment}
\begin{eulerprompt}
>aspect(1.5); x=1:10; y=-log(random(size(x)))*200; ...
>logimpulseplot1(x,y):
\end{eulerprompt}
\eulerimg{17}{images/22305144006-EMT4Plot2D-156.png}
\begin{eulercomment}
Mari kita menghidupkan kurva 2D dengan menggunakan plot langsung.
Perintah plot(x,y) hanya memplot kurva ke dalam jendela plot.
setplot(a,b,c,d) mengatur jendela ini.

Fungsi wait(0) memaksa plot untuk muncul pada jendela grafik. Jika
tidak, penggambaran ulang akan dilakukan dalam interval waktu yang
jarang.
\end{eulercomment}
\begin{eulerprompt}
>function animliss (n,m) ...
\end{eulerprompt}
\begin{eulerudf}
  t=linspace(0,2pi,500);
  f=0;
  c=framecolor(0);
  l=linewidth(2);
  setplot(-1,1,-1,1);
  repeat
    clg;
    plot(sin(n*t),cos(m*t+f));
    wait(0);
    if testkey() then break; endif;
    f=f+0.02;
  end;
  framecolor(c);
  linewidth(l);
  endfunction
\end{eulerudf}
\begin{eulercomment}
Press any key to stop this animation.
\end{eulercomment}
\begin{eulerprompt}
>animliss(2,3); // lihat hasilnya, jika sudah puas, tekan ENTER
\end{eulerprompt}
\eulerheading{Plot Logaritmik}
\begin{eulercomment}
EMT menggunakan parameter "logplot" untuk skala logaritmik.\\
Plot logaritmik dapat diplot menggunakan skala logaritmik dalam y
dengan logplot = 1, atau menggunakan skala logaritmik dalam x dan y
dengan logplot = 2, atau dalam x dengan logplot = 3.

\end{eulercomment}
\begin{eulerttcomment}
 - logplot = 1: y-logaritmik
 - logplot=2: x-y-logaritmik
 - logplot=3: x-logaritmik
\end{eulerttcomment}
\begin{eulerprompt}
>plot2d("exp(x^3-x)*x^2",1,5,logplot=1):
\end{eulerprompt}
\eulerimg{17}{images/22305144006-EMT4Plot2D-157.png}
\begin{eulerprompt}
>plot2d("exp(x+sin(x))",0,100,logplot=1):
\end{eulerprompt}
\eulerimg{17}{images/22305144006-EMT4Plot2D-158.png}
\begin{eulerprompt}
>plot2d("exp(x+sin(x))",10,100,logplot=2):
\end{eulerprompt}
\eulerimg{17}{images/22305144006-EMT4Plot2D-159.png}
\begin{eulerprompt}
>plot2d("gamma(x)",1,10,logplot=1):
\end{eulerprompt}
\eulerimg{17}{images/22305144006-EMT4Plot2D-160.png}
\begin{eulerprompt}
>plot2d("log(x*(2+sin(x/100)))",10,1000,logplot=3):
\end{eulerprompt}
\eulerimg{17}{images/22305144006-EMT4Plot2D-161.png}
\begin{eulercomment}
Hal ini juga bisa dilakukan dengan plot data.
\end{eulercomment}
\begin{eulerprompt}
>x=10^(1:20); y=x^2-x;
>plot2d(x,y,logplot=2):
\end{eulerprompt}
\eulerimg{17}{images/22305144006-EMT4Plot2D-162.png}
\begin{eulerprompt}
>     
\end{eulerprompt}
\begin{euleroutput}
  
\end{euleroutput}
\eulerheading{Contoh Soal}
\begin{eulercomment}
\end{eulercomment}
\eulersubheading{Program Linear}
\begin{eulercomment}
Suatu perusahaan yang mempunyai 3 pabrik akan memproduksi 2 jenis
produk. Pabrik 1 dapat menghasilkan satu unit produk I selama 1 jam
dan produk II selama 2 jam. Pabrik 2 dapat menghasilkan satu unit
produk I selama 1 jam dan menghasilkan produk II selama 1 jam . Pabrik
3 dapat menghasilkan satu unit produk I selama 3 jam dan menghasilkan
satu unit produk II selama 2 jam. Kapasitas produksi pabrik 1 setiap
pekannya adalah dapat beroperasi paling lama 10 jam, pabrik 2 paling
lama 6 jam, dan pabrik 3 paling lama 16 jam. Adapun keuntungan produk
I per unit adalah 3 dan produk II adalah 5. Perusahaan tersebut ingin
memaksimumkan laba yang diperoleh dengan keterbatasan kapasitas
produksi setiap pabriknya. Masalahnya adalah berapa unit masing-masing
produk I dan produk II yang harus diproduksi

Kemudian dijadikan model matematika\\
-fungsi tujuan:\\
\end{eulercomment}
\begin{eulerformula}
\[
f(x,y)=3x+5y
\]
\end{eulerformula}
\begin{eulercomment}
-Kendala\\
\end{eulercomment}
\begin{eulerformula}
\[
x+2y
\]
\end{eulerformula}
\begin{eulerformula}
\[
x+y
\]
\end{eulerformula}
\begin{eulerformula}
\[
3x+2y
\]
\end{eulerformula}
\begin{eulercomment}
\end{eulercomment}
\begin{eulerformula}
\[
\text{Pabrik 1 dapat beroperasi}\le 10 \text{jam}
\]
\end{eulerformula}
\begin{eulerformula}
\[
\text{Pabrik 2 dapat beroperasi}\le 6 \text{jam}
\]
\end{eulerformula}
\begin{eulerformula}
\[
\text{Pabrik 3 dapat beroperasi}\le 16 \text{jam}
\]
\end{eulerformula}
\begin{eulercomment}
dan diperoleh kendala:\\
\end{eulercomment}
\begin{eulerformula}
\[
x+2y\le10
\]
\end{eulerformula}
\begin{eulerformula}
\[
x+y\le 6
\]
\end{eulerformula}
\begin{eulerformula}
\[
3x+2y\le16
\]
\end{eulerformula}
\begin{eulerformula}
\[
x\ge 0
\]
\end{eulerformula}
\begin{eulerformula}
\[
y\ge 0
\]
\end{eulerformula}
\begin{eulercomment}
Kemudian akan digambar grafik
\end{eulercomment}
\begin{eulerprompt}
> plot2d(["(10-x)/2","(6-x)","(3*x-16)/-2"] ...
>,0,10,0,10,color=13:15,thickness=3,grid=5,):
\end{eulerprompt}
\eulerimg{27}{images/22305144006-EMT4Plot2D-175.png}
\begin{eulercomment}
Mencari perpotongan bisa dengan menggunakan solve

Kemudian kendala dibuat kedalam bentuk matriks koefisien
\end{eulercomment}
\begin{eulerprompt}
>A=[1,2;1,1;3,2]; b=[10;6;16]; xa=feasibleArea(A,b);
>x=simplex(A,b,[3,5],>max); fraction x
\end{eulerprompt}
\begin{euleroutput}
          2 
          4 
\end{euleroutput}
\begin{eulerprompt}
>plot2d(xa[1],xa[2],>filled,style="/",a=0,b=10,c=0,d=10,>add);
>plot2d(x[1],x[2],>add,>points):
\end{eulerprompt}
\eulerimg{27}{images/22305144006-EMT4Plot2D-176.png}
\begin{eulerprompt}
>    
\end{eulerprompt}
\begin{euleroutput}
  
\end{euleroutput}
\begin{eulercomment}
Kemudian tinggal di substitusikan saja titik (2,4) ke dalam fungsi
tujuan maka ditemukan nilai optimal

\end{eulercomment}
\begin{eulerformula}
\[
3x+5y
\]
\end{eulerformula}
\begin{eulerformula}
\[
3*2+5*4
\]
\end{eulerformula}
\begin{eulerformula}
\[
26
\]
\end{eulerformula}
\begin{eulercomment}
Jadi maksimal pabrik memperoleh laba adalah 26 dan produk 1 dibuat
sebanyak 2 dan produk 2 dibuat sebnayak 4

\end{eulercomment}
\eulersubheading{Kalkulus Multivariabel}
\begin{eulercomment}
Buatlah gagasan visual tentang bagaimana polinomial taylor memberikan
aproximasi terhadap cos x
\end{eulercomment}
\begin{eulerprompt}
> plot2d("cos(x)",a=-5,b=5,c=-1,d=2,color=green,thickness=2,grid=1); ...
>  plot2d(&taylor(cos(x),x,0,0),>add,color=blue); ...
>  plot2d(&taylor(cos(x),x,0,2),>add,color=red); ...
>  plot2d(&taylor(cos(x),x,0,4),>add,color=orange); ...
> plot2d(&taylor(cos(x),x,0,8),>add,color=black);...
>  labelbox(["cos(x)","T1","T2","T4","T8"],styles=["-","-","-","-","-"], ...
>    colors=[green,blue,red,orange,black]):
\end{eulerprompt}
\eulerimg{27}{images/22305144006-EMT4Plot2D-180.png}
\begin{eulercomment}
2. Nyatakan grafik persamaan polar berikut\\
\end{eulercomment}
\begin{eulerformula}
\[
r=sin3\Theta+sin^2(2\Theta)
\]
\end{eulerformula}
\begin{eulerprompt}
>t=linspace(0,2pi,1000); r=sin(3*t)+(sin(2*t))^2; x=r*cos(t); y=r*sin(t); ...
>plot2d(x,y,>filled,fillcolor=red,style="/",r=1.5):
\end{eulerprompt}
\eulerimg{27}{images/22305144006-EMT4Plot2D-182.png}
\begin{eulercomment}
3. Nyatakan grafik persamaan polar berikut\\
\end{eulercomment}
\begin{eulerformula}
\[
r=cos2\Theta+cos^2(4\Theta)
\]
\end{eulerformula}
\begin{eulerprompt}
> t=linspace(0,2pi,1000); r=cos(2*t)+(cos(4*t))^2; x=r*cos(t); y=r*sin(t); ...
>plot2d(x,y,>filled,fillcolor=red,style="/",r=2):
\end{eulerprompt}
\eulerimg{27}{images/22305144006-EMT4Plot2D-184.png}
\begin{eulercomment}
4. Nyatakan grafik persamaan polar berikut\\
\end{eulercomment}
\begin{eulerformula}
\[
r=sin4\Theta+sin^2(5\Theta)
\]
\end{eulerformula}
\begin{eulerprompt}
> t=linspace(0,2pi,1000); r=sin(4*t)+(sin(5*t))^2; x=r*cos(t); y=r*sin(t); ...
>plot2d(x,y,>filled,fillcolor=red,style="/",r=2):
\end{eulerprompt}
\eulerimg{27}{images/22305144006-EMT4Plot2D-186.png}
\begin{eulercomment}
5. Nyatakan grafik persamaan polar berikut\\
\end{eulercomment}
\begin{eulerformula}
\[
r=cos2\Theta+cos^2(3\Theta)
\]
\end{eulerformula}
\begin{eulerprompt}
> t=linspace(0,2pi,1000); r=cos(2*t)+(cos(3*t))^2; x=r*cos(t); y=r*sin(t); ...
>plot2d(x,y,>filled,fillcolor=red,style="/",r=2):
\end{eulerprompt}
\eulerimg{27}{images/22305144006-EMT4Plot2D-188.png}
\begin{eulerprompt}
>t=linspace(0,2pi,1000); r=4*cos(2*t); x=r*cos(t); y=r*sin(t); ...
>plot2d(x,y,>filled,fillcolor=red,style="O");
> t=linspace(0,2pi,1000); r=4*sin(2*t); x=r*cos(t); y=r*sin(t); ...
>plot2d(x,y,>filled,fillcolor=green,style="O",r=4,>add);  
>  t=linspace(0,2pi,1000); r=cos(4*t)+(cos(4*t))^2; x=r*cos(t); y=r*sin(t); ...
>plot2d(x,y,>filled,fillcolor=yellow,style="O",>add):
\end{eulerprompt}
\eulerimg{27}{images/22305144006-EMT4Plot2D-189.png}
\begin{eulercomment}
7. Nyatakan grafik persamaan polar berikut\\
\end{eulercomment}
\begin{eulerformula}
\[
r=sin4\Theta+sin^2(4\Theta)
\]
\end{eulerformula}
\begin{eulerprompt}
> t=linspace(0,2pi,1000); r=sin(4*t)+(sin(4*t))^2; x=r*cos(t); y=r*sin(t); ...
>plot2d(x,y,>filled,fillcolor=red,style="/",r=2):
\end{eulerprompt}
\eulerimg{27}{images/22305144006-EMT4Plot2D-191.png}
\eulersubheading{Statistika}
\begin{eulercomment}
Hubungan antara kompetensi (X) dan kinerja pegawai (Y) kita ambil
sampel acak 10 orang pegawai sebagai berikut

Akan ditentukan grafik persamaan regresi Y atas X maka didapat:
\end{eulercomment}
\begin{eulerprompt}
>xdata=[22,30,32,36,50,52,54,59,60,61]; ...
>ydata=[4,16,12,24,15,24,22,17,19,14]; // data
>plot2d(xdata,ydata,a=20,b=70,c=0,d=30,style="."); // lines
>plot2d(xdata,ydata,>points,>add,style="o"): // add points
\end{eulerprompt}
\eulerimg{27}{images/22305144006-EMT4Plot2D-192.png}
\begin{eulerprompt}
>p=polyfit(xdata,ydata,1); // get regression line
>plot2d("polyval(p,x)",>add,color=red): // add plot of line
\end{eulerprompt}
\eulerimg{27}{images/22305144006-EMT4Plot2D-193.png}
\eulersubheading{Soal Random}
\begin{eulercomment}
1. Buatlah plot x\textasciicircum{}2 dengan interval atau n (1 sampai 10)
\end{eulercomment}
\begin{eulerprompt}
>plot2d("x^2",0,1,steps=1,color=1,n=1); plot2d("x^2",>add,steps=2,color=2,n=2); ...
>plot2d("x^2",>add,steps=3,color=3,n=3); plot2d("x^2",>add,steps=4,color=4,n=4); ...
>plot2d("x^2",>add,steps=5,color=5,n=5); plot2d("x^2",>add,steps=6,color=6,n=6); ...
>plot2d("x^2",>add,steps=7,color=7,n=7); plot2d("x^2",>add,steps=8,color=8,n=8); ...
>plot2d("x^2",>add,steps=9,color=9,n=9); plot2d("x^2",>add,steps=10,color=10,n=10):
\end{eulerprompt}
\eulerimg{27}{images/22305144006-EMT4Plot2D-194.png}
\begin{eulercomment}
2. Gambar kurva x\textasciicircum{}i dengan i adalah bilangan asli 1 sampai 5
\end{eulercomment}
\begin{eulerprompt}
>plot2d(["x","x^2","x^3","x^4","x^5"],a=-10,b=10,c=-10,d=10,color=10:14):
\end{eulerprompt}
\eulerimg{27}{images/22305144006-EMT4Plot2D-195.png}
\eulerheading{Rujukan Lengkap Fungsi plot2d()}
\begin{eulercomment}
\end{eulercomment}
\begin{eulerttcomment}
  function plot2d (xv, yv, btest, a, b, c, d, xmin, xmax, r, n,  ..
  logplot, grid, frame, framecolor, square, color, thickness, style, ..
  auto, add, user, delta, points, addpoints, pointstyle, bar, histogram,  ..
  distribution, even, steps, own, adaptive, hue, level, contour,  ..
  nc, filled, fillcolor, outline, title, xl, yl, maps, contourcolor, ..
  contourwidth, ticks, margin, clipping, cx, cy, insimg, spectral,  ..
  cgrid, vertical, smaller, dl, niveau, levels)
\end{eulerttcomment}
\begin{eulercomment}
Multipurpose plot function for plots in the plane (2D plots). This function can do
plots of functions of one variables, data plots, curves in the plane, bar plots, grids
of complex numbers, and implicit plots of functions of two variables.

Parameters
\\
x,y       : equations, functions or data vectors\\
a,b,c,d   : Plot area (default a=-2,b=2)\\
r         : if r is set, then a=cx-r, b=cx+r, c=cy-r, d=cy+r\\
\end{eulercomment}
\begin{eulerttcomment}
            r can be a vector [rx,ry] or a vector [rx1,rx2,ry1,ry2].
\end{eulerttcomment}
\begin{eulercomment}
xmin,xmax : range of the parameter for curves\\
auto      : Determine y-range automatically (default)\\
square    : if true, try to keep square x-y-ranges\\
n         : number of intervals (default is adaptive)\\
grid      : 0 = no grid and labels,\\
\end{eulercomment}
\begin{eulerttcomment}
            1 = axis only,
            2 = normal grid (see below for the number of grid lines)
            3 = inside axis
            4 = no grid
            5 = full grid including margin
            6 = ticks at the frame
            7 = axis only
            8 = axis only, sub-ticks
\end{eulerttcomment}
\begin{eulercomment}
frame     : 0 = no frame\\
framecolor: color of the frame and the grid\\
margin    : number between 0 and 0.4 for the margin around the plot\\
color     : Color of curves. If this is a vector of colors,\\
\end{eulercomment}
\begin{eulerttcomment}
            it will be used for each row of a matrix of plots. In the case of
            point plots, it should be a column vector. If a row vector or a
            full matrix of colors is used for point plots, it will be used for
            each data point.
\end{eulerttcomment}
\begin{eulercomment}
thickness : line thickness for curves\\
\end{eulercomment}
\begin{eulerttcomment}
            This value can be smaller than 1 for very thin lines.
\end{eulerttcomment}
\begin{eulercomment}
style     : Plot style for lines, markers, and fills.\\
\end{eulercomment}
\begin{eulerttcomment}
            For points use
            "[]", "<>", ".", "..", "...",
            "*", "+", "|", "-", "o"
            "[]#", "<>#", "o#" (filled shapes)
            "[]w", "<>w", "ow" (non-transparent)
            For lines use
            "-", "--", "-.", ".", ".-.", "-.-", "->"
            For filled polygons or bar plots use
            "#", "#O", "O", "/", "\(\backslash\)", "\(\backslash\)/",
            "+", "|", "-", "t"
\end{eulerttcomment}
\begin{eulercomment}
points    : plot single points instead of line segments\\
addpoints : if true, plots line segments and points\\
add       : add the plot to the existing plot\\
user      : enable user interaction for functions\\
delta     : step size for user interaction\\
bar       : bar plot (x are the interval bounds, y the interval values)\\
histogram : plots the frequencies of x in n subintervals\\
distribution=n : plots the distribution of x with n subintervals\\
even      : use inter values for automatic histograms.\\
steps     : plots the function as a step function (steps=1,2)\\
adaptive  : use adaptive plots (n is the minimal number of steps)\\
level     : plot level lines of an implicit function of two variables\\
outline   : draws boundary of level ranges.
\\
If the level value is a 2xn matrix, ranges of levels will be drawn\\
in the color using the given fill style. If outline is true, it\\
will be drawn in the contour color. Using this feature, regions of\\
f(x,y) between limits can be marked.
\\
hue       : add hue color to the level plot to indicate the function\\
\end{eulercomment}
\begin{eulerttcomment}
            value
\end{eulerttcomment}
\begin{eulercomment}
contour   : Use level plot with automatic levels\\
nc        : number of automatic level lines\\
title     : plot title (default "")\\
xl, yl    : labels for the x- and y-axis\\
smaller   : if \textgreater{}0, there will be more space to the left for labels.\\
vertical  :\\
\end{eulercomment}
\begin{eulerttcomment}
  Turns vertical labels on or off. This changes the global variable
  verticallabels locally for one plot. The value 1 sets only vertical
  text, the value 2 uses vertical numerical labels on the y axis.
\end{eulerttcomment}
\begin{eulercomment}
filled    : fill the plot of a curve\\
fillcolor : fill color for bar and filled curves\\
outline   : boundary for filled polygons\\
logplot   : set logarithmic plots\\
\end{eulercomment}
\begin{eulerttcomment}
            1 = logplot in y,
            2 = logplot in xy,
            3 = logplot in x
\end{eulerttcomment}
\begin{eulercomment}
own       :\\
\end{eulercomment}
\begin{eulerttcomment}
  A string, which points to an own plot routine. With >user, you get
  the same user interaction as in plot2d. The range will be set
  before each call to your function.
\end{eulerttcomment}
\begin{eulercomment}
maps      : map expressions (0 is faster), functions are always mapped.\\
contourcolor : color of contour lines\\
contourwidth : width of contour lines\\
clipping  : toggles the clipping (default is true)\\
title     :\\
\end{eulercomment}
\begin{eulerttcomment}
  This can be used to describe the plot. The title will appear above
  the plot. Moreover, a label for the x and y axis can be added with
  xl="string" or yl="string". Other labels can be added with the
  functions label() or labelbox(). The title can be a unicode
  string or an image of a Latex formula.
\end{eulerttcomment}
\begin{eulercomment}
cgrid     :\\
\end{eulercomment}
\begin{eulerttcomment}
  Determines the number of grid lines for plots of complex grids.
  Should be a divisor of the the matrix size minus 1 (number of
  subintervals). cgrid can be a vector [cx,cy].
\end{eulerttcomment}
\begin{eulercomment}

Overview

The function can plot

- expressions, call collections or functions of one variable,\\
- parametric curves,\\
- x data against y data,\\
- implicit functions,\\
- bar plots,\\
- complex grids,\\
- polygons.

If a function or expression for xv is given, plot2d() will compute\\
values in the given range using the function or expression. The\\
expression must be an expression in the variable x. The range must\\
be defined in the parameters a and b unless the default range\\
[-2,2] should be used. The y-range will be computed automatically,\\
unless c and d are specified, or a radius r, which yields the range\\
[-r,r] for x and y. For plots of functions, plot2d will use an\\
adaptive evaluation of the function by default. To speed up the\\
plot for complicated functions, switch this off with \textless{}adaptive, and\\
optionally decrease the number of intervals n. Moreover, plot2d()\\
will by default use mapping. I.e., it will compute the plot element\\
for element. If your expression or your functions can handle a\\
vector x, you can switch that off with \textless{}maps for faster evaluation.

Note that adaptive plots are always computed element for element. \\
If functions or expressions for both xv and for yv are specified,\\
plot2d() will compute a curve with the xv values as x-coordinates\\
and the yv values as y-coordinates. In this case, a range should be\\
defined for the parameter using xmin, xmax. Expressions contained\\
in strings must always be expressions in the parameter variable x.
\end{eulercomment}
\end{eulernotebook}
